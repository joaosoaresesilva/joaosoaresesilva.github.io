\documentclass[12pt]{article}
\usepackage[utf8]{inputenc}
\usepackage[portuguese]{babel}
\usepackage{geometry}
\geometry{a4paper, margin=1in}
\usepackage{amsmath}
\usepackage{amsfonts}
\usepackage{amssymb}
\usepackage{graphicx}
\usepackage{enumitem} % Para personalizar listas e dar mais controlo sobre o espaçamento
\usepackage{titlesec} % Para personalizar títulos

% Configurações de títulos para um visual mais limpo e profissional
\titleformat{\section}[block]{\bfseries\Large\raggedright}{}{0em}{}
\titlespacing{\section}{0pt}{1.5em}{1em}

\title{Plano de Formação: Linux - Serviços de Redes}
\author{Formador: [Seu Nome]} % Altere aqui para o seu nome
\date{Data de Elaboração: [Data]} % Altere aqui a data

\begin{document}
	
	% --- Cabeçalho do Documento ---
	\begin{center}
		\vspace*{2cm}
		{\Huge\bfseries Plano de Formação} \\
		\vspace{0.5cm}
		{\Large\itshape Linux - Serviços de Redes}
		\vspace{2cm}
	\end{center}
	
	\begin{itemize}[leftmargin=*,labelsep=1cm]
		\item \textbf{Designação da UFCD:} Linux - serviços de redes
		\item \textbf{Código da UFCD:} 0839
		\item \textbf{Carga Horária:} 50 horas
		\item \textbf{Pontos de Crédito:} 4.5
	\end{itemize}
	
	\vspace{1cm}
	
	% --- Objetivos de Formação ---
	\section*{Objetivos}
	\begin{itemize}
		\item Instalar e configurar o linux server - serviços de redes.
		\item Instalar e configurar o linux server - NIS.
		\item Instalar e configurar o linux server - DHCP.
		\item Instalar e configurar o linux server - DNS.
		\item Instalar e configurar o linux server - LOGS.
	\end{itemize}
	
	\newpage
	
	% --- Conteúdos Programáticos ---
	\section*{Conteúdos}
	
	\subsection*{1. Serviços de rede}
	\begin{itemize}
		\item \texttt{/etc/rc.d/init.d/}
		\item Iniciação e paragem dos serviços
		\item Pasta \texttt{/etc/services}
		\item Lista de portas e serviços no Linux
		\item Encerramento de um serviço ou porta
	\end{itemize}
	
	\subsection*{2. XINET.d}
	\begin{itemize}
		\item Arquivo \texttt{/etc/xinetd.conf}
		\item Pasta \texttt{/etc/xinet.d/}
	\end{itemize}
	
	\subsection*{3. TCPWrappers}
	\begin{itemize}
		\item \texttt{/etc/hosts.allow}
		\item \texttt{/etc/hosts.deny}
	\end{itemize}
	
	\subsection*{4. NIS}
	\begin{itemize}
		\item Configuração de um servidor NIS (Network Information Service)
		\item Criação de um domínio NIS
		\item Arquivo \texttt{/etc/yp.conf}
		\item Configuração de um Cliente NIS
		\item Acesso a contas no NIS
	\end{itemize}
	
	\subsection*{5. DHCP}
	\begin{itemize}
		\item Conceito
		\item Iniciação do servidor DHCP
		\item Descrição dos principais parâmetros:
		\begin{itemize}
			\item lease time
			\item range
			\item mac address
			\item routers
			\item domain name
			\item name servers
		\end{itemize}
		\item Arquivo \texttt{/var/lib/dhcp/dhcpd.leases}
		\item Configuração do range de uma rede
		\item Definição de informações para a rede TCP
		\item Definição de IP e informações para uma máquina específica na rede através de seu endereço físico
		\item Definição de IPs para todas as máquinas na rede através de seu endereço físico
		\item Coexistência de mais de um servidor DHCP na rede
		\item Configuração de um cliente para acesso à rede DHCP
		\item Comando \texttt{pump}
		\item DHCP do Linux
	\end{itemize}
	
	\subsection*{6. DNS}
	\begin{itemize}
		\item Conceitos
		\begin{itemize}
			\item Zona
			\item Domínios
			\item Nós
			\item Servidores Matriz (root servers)
		\end{itemize}
		\item FAPESP e Internic
		\item DNS e replicação de zonas
		\item BIND (\texttt{named}) - Berkeley Internet Name Domain
		\item Arquivo \texttt{/etc/named.conf}
		\item Instruções \texttt{options} e \texttt{zone}
		\item Arquivo \texttt{/var/named/named.ca}
		\item Criação e edição de zonas
		\item Delegação de autorização para novas zonas
		\item Configuração da replicação das zonas
		\item Configuração de um servidor Master e Slave
		\item Iniciação do servidor DNS
		\item Papel do DNS e do \texttt{hosts}
		\item Configuração de um cliente
		\item Máquinas a inserir no DNS
	\end{itemize}
	
	\subsection*{7. LOGS}
	\begin{itemize}
		\item Arquivos de log do sistema
		\item Pasta \texttt{/var/log}
		\item Arquivo \texttt{messages}
		\item Syslogd
		\item Arquivo \texttt{syslog}
		\item Outros arquivos de log de aplicativos
		\begin{itemize}
			\item Apache
			\item Sendmail
		\end{itemize}
	\end{itemize}
	
\end{document}