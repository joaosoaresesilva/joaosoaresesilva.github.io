\documentclass[12pt]{article}
\usepackage[utf8]{inputenc}
\usepackage{geometry}
\usepackage{graphicx}
\usepackage{hyperref}
\usepackage{enumitem}
\usepackage{booktabs}

\geometry{a4paper, margin=2.5cm}
\title{Plano Sessão \\ \large Data Science Aplicada  Um Guia para Análise, Modelação e ComunicaçãoR}
\author{Formador: JOÃO SILVA}
\date{}

\begin{document}

\maketitle



\section*{Objetivo Geral}
Capacitar os formandos para atuar em equipas de ciência de dados, com domínio das ferramentas em R para coleta, análise, modelação e comunicação de dados.

\section*{Destinatários}
Formandos com conhecimentos básicos em programação e estatística.

\section*{Carga Horária}
60 horas, distribuídas em 7 módulos.

\section*{Introdução}



\noindent\textbf{Data Collection and Preparation:} Assisting in the collection, cleaning, and preprocessing of large datasets from various sources, ensuring data quality and integrity.

\noindent\textbf{Exploratory Data Analysis (EDA):} Performing initial data exploration and visualization to identify patterns, trends, and anomalies, and communicating findings to the team.

\noindent\textbf{Model Development Support:} Collaborating with senior colleagues on the development and implementation of machine learning models, including feature engineering, model training, and evaluation.

\noindent\textbf{Reporting and Documentation:} Generating reports, dashboards, and presentations to effectively communicate data insights and model performance to technical and non-technical stakeholders.

\noindent\textbf{Research and Learning:} Staying up-to-date with the latest advancements in data science, machine learning, and artificial intelligence, and continuously improving your skills.

\noindent\textbf{Ad-hoc Analysis:} Conducting ad-hoc data analysis to answer specific business questions and provide data-driven recommendations.

\noindent\textbf{Tooling and Infrastructure:} Assisting in the maintenance and optimization of data science tools and infrastructure.

\vspace{0.5cm}

\section*{Introdução}

Este curso  foi concebido para preparar formandos que  desempenham funções essenciais no ciclo de vida de projetos de Ciência de Dados, abrangendo desde a recolha e preparação de dados até à implementação e manutenção de modelos preditivos. 

\noindent\textbf{Recolha e Preparação de Dados:} Apoiar na recolha, limpeza e pré-processamento de grandes conjuntos de dados provenientes de várias fontes, garantindo a qualidade e a integridade dos dados.

\noindent\textbf{Análise Exploratória de Dados (EDA):} Realizar a exploração inicial e a visualização de dados para identificar padrões, tendências e anomalias, comunicando as conclusões à equipa.

\noindent\textbf{Apoio ao Desenvolvimento de Modelos:} Colaborar com colegas séniores no desenvolvimento e implementação de modelos de aprendizagem automática, incluindo engenharia de variáveis, treino e avaliação de modelos.

\noindent\textbf{Relatórios e Documentação:} Produzir relatórios, \textit{dashboards} e apresentações para comunicar de forma eficaz as perceções obtidas a partir dos dados e o desempenho dos modelos, tanto a públicos técnicos como não técnicos.

\noindent\textbf{Investigação e Aprendizagem:} Manter-se atualizado relativamente aos mais recentes avanços em ciência de dados, aprendizagem automática e inteligência artificial, melhorando continuamente as suas competências.

\noindent\textbf{Análise \textit{Ad-hoc}:} Realizar análises de dados pontuais para responder a questões específicas de negócio e fornecer recomendações baseadas em dados.

\noindent\textbf{Ferramentas e Infraestrutura:} Apoiar na manutenção e otimização das ferramentas e da infraestrutura de ciência de dados.

\vspace{0.5cm}


\subsection*{Resumo}
Este curso fornece uma visão prática e integrada do trabalho em Ciência de Dados com R. Os estudantes irão:
\begin{itemize}
    \item Aprender a recolher, limpar e preparar dados de múltiplas fontes, garantindo qualidade e integridade.
    \item Explorar dados com técnicas de análise exploratória e visualização para identificar padrões e anomalias.
    \item Apoiar no desenvolvimento de modelos de \textit{machine learning}, desde a engenharia de variáveis até à avaliação.
    \item Criar relatórios, dashboards e apresentações para comunicar resultados a públicos técnicos e não técnicos.
    \item Manter-se atualizados com as tendências e avanços na área.
    \item Realizar análises pontuais para responder a questões específicas de negócio.
    \item Utilizar e otimizar ferramentas e infraestruturas de suporte à Ciência de Dados.
\end{itemize}


\section*{Módulos de Formação}

%\subsection*{Módulo 1 – Data Collection and Preparation (10h)}
%\textbf{Objetivos:}
%\begin{itemize}
%  \item Importar dados de diversas fontes (CSV, Excel, APIs, bases SQL)
%  \item Limpar e transformar dados com \texttt{dplyr}, \texttt{tidyr}, \texttt{janitor}
%  \item Garantir integridade e qualidade dos dados
%\end{itemize}
%
%\textbf{Atividades Práticas:}
%\begin{itemize}
%  \item Importar e limpar um dataset do Kaggle ou de uma API pública
%  \item Criar uma função de limpeza automatizada
%\end{itemize}

\subsection*{Módulo 1 – Data Collection and Preparation (10h)}

\textbf{Objetivos:}
\begin{itemize}
  \item Compreender a importância da recolha e preparação de dados no ciclo de vida de um projeto de Data Science.
  \item Importar dados de diversas fontes, incluindo ficheiros locais (\texttt{CSV}, \texttt{Excel}), APIs REST, e bases de dados relacionais (\texttt{MySQL}, \texttt{PostgreSQL}, \texttt{SQLite}).
  \item Aplicar técnicas de limpeza e transformação de dados utilizando pacotes do ecossistema \texttt{tidyverse} (\texttt{dplyr}, \texttt{tidyr}) e funções auxiliares do pacote \texttt{janitor}.
  \item Garantir a integridade e qualidade dos dados através de verificação de tipos, deteção de valores ausentes, duplicados e inconsistências.
  \item Documentar o processo de recolha e preparação para assegurar reprodutibilidade e transparência.
\end{itemize}

\textbf{Conteúdos:}
\begin{itemize}
  \item Introdução ao processo de \textit{Data Collection} e \textit{Data Preparation}.
  \item Leitura de dados:
    \begin{itemize}
      \item Ficheiros \texttt{CSV} e \texttt{Excel} com \texttt{readr} e \texttt{readxl}.
      \item Acesso a APIs com \texttt{httr} e \texttt{jsonlite}.
      \item Conexão a bases de dados com \texttt{DBI} e \texttt{RSQLite}/\texttt{RMySQL}.
    \end{itemize}
  \item Limpeza de dados:
    \begin{itemize}
      \item Normalização de nomes de colunas com \texttt{janitor::clean\_names()}.
      \item Tratamento de valores ausentes (\texttt{NA}) e duplicados.
      \item Conversão de tipos de dados (datas, fatores, numéricos).
    \end{itemize}
  \item Transformação de dados:
    \begin{itemize}
      \item Filtragem, ordenação e seleção de variáveis.
      \item Criação de variáveis derivadas.
      \item \textit{Reshaping} de dados com \texttt{pivot\_longer()} e \texttt{pivot\_wider()}.
    \end{itemize}
  \item Boas práticas de organização e documentação do código.
\end{itemize}

\textbf{Atividades Práticas:}
\begin{itemize}
  \item Importar e limpar um dataset real obtido do Kaggle ou de uma API pública (por exemplo, dados meteorológicos ou de vendas).
  \item Criar uma função personalizada em R para automatizar tarefas de limpeza recorrentes (remoção de duplicados, normalização de nomes, tratamento de \texttt{NA}).
  \item Validar a integridade dos dados através de verificações automáticas (contagem de registos, tipos de variáveis, intervalos de valores).
  \item Documentar o processo de recolha e preparação num ficheiro \texttt{RMarkdown}, incluindo código, explicações e exemplos de antes/depois da limpeza.
\end{itemize}

\textbf{Recursos e Ferramentas:}
\begin{itemize}
  \item R e RStudio.
  \item Pacotes: \texttt{tidyverse}, \texttt{janitor}, \texttt{httr}, \texttt{jsonlite}, \texttt{DBI}.
  \item Fontes de dados: Kaggle, APIs públicas (ex.: OpenWeatherMap, World Bank), bases de dados de teste.
\end{itemize}





%\subsection*{Módulo 2 – Exploratory Data Analysis (EDA) (10h)}
%\textbf{Objetivos:}
%\begin{itemize}
%  \item Explorar dados com estatísticas descritivas e visualizações
%  \item Identificar padrões, outliers e correlações
%\end{itemize}
%
%\textbf{Ferramentas:} \texttt{ggplot2}, \texttt{plotly}, \texttt{corrplot}


\subsection*{Módulo 2 – Exploratory Data Analysis (EDA) (10h)}

\textbf{Objetivos:}
\begin{itemize}
  \item Compreender o papel da Análise Exploratória de Dados no ciclo de vida de um projeto de Data Science.
  \item Explorar dados utilizando estatísticas descritivas e visualizações gráficas para obter uma compreensão inicial do conjunto de dados.
  \item Identificar padrões, tendências, outliers e relações entre variáveis.
  \item Formular hipóteses iniciais com base nas observações obtidas.
  \item Comunicar de forma clara os principais achados da análise exploratória.
\end{itemize}

\textbf{Conteúdos:}
\begin{itemize}
  \item Introdução à Análise Exploratória de Dados (EDA) e sua importância.
  \item Estatísticas descritivas:
    \begin{itemize}
      \item Medidas de tendência central: média, mediana, moda.
      \item Medidas de dispersão: variância, desvio padrão, amplitude.
      \item Distribuições de frequência e percentis.
    \end{itemize}
  \item Visualização de dados:
    \begin{itemize}
      \item Gráficos básicos com \texttt{ggplot2}: histogramas, boxplots, scatterplots, gráficos de barras.
      \item Visualizações interativas com \texttt{plotly}.
      \item Mapas de calor e matrizes de correlação com \texttt{corrplot}.
    \end{itemize}
  \item Identificação de outliers e valores atípicos:
    \begin{itemize}
      \item Métodos gráficos (boxplot, scatterplot).
      \item Métodos estatísticos (IQR, Z-score).
    \end{itemize}
  \item Análise de correlação:
    \begin{itemize}
      \item Correlação de Pearson, Spearman e Kendall.
      \item Interpretação de coeficientes de correlação.
    \end{itemize}
  \item Boas práticas na apresentação de resultados da EDA.
\end{itemize}

\textbf{Atividades Práticas:}
\begin{itemize}
  \item Calcular estatísticas descritivas para um dataset real (ex.: vendas, saúde pública, meteorologia).
  \item Criar visualizações com \texttt{ggplot2} para explorar distribuições e relações entre variáveis.
  \item Utilizar \texttt{plotly} para criar gráficos interativos que permitam explorar dados dinamicamente.
  \item Gerar e interpretar uma matriz de correlação com \texttt{corrplot}.
  \item Identificar e documentar padrões, outliers e correlações relevantes.
\end{itemize}

\textbf{Recursos e Ferramentas:}
\begin{itemize}
  \item R e RStudio.
  \item Pacotes: \texttt{ggplot2}, \texttt{plotly}, \texttt{corrplot}, \texttt{dplyr}.
  \item Datasets de apoio: vendas online, dados de saúde pública, dados meteorológicos.
\end{itemize}



%\subsection*{Módulo 3 – Model Development Support (12h)}
%\textbf{Objetivos:}
%\begin{itemize}
%  \item Apoiar na criação de modelos preditivos
%  \item Realizar engenharia de variáveis e avaliação de modelos
%\end{itemize}
%
%\textbf{Ferramentas:} \texttt{caret}, \texttt{tidymodels}, \texttt{recipes}


\subsection*{Módulo 3 – Model Development Support (12h)}

\textbf{Objetivos:}
\begin{itemize}
  \item Compreender o papel do desenvolvimento de modelos no ciclo de vida de um projeto de Data Science.
  \item Apoiar na criação de modelos preditivos supervisionados e não supervisionados.
  \item Realizar engenharia de variáveis (\textit{feature engineering}) para melhorar a performance dos modelos.
  \item Selecionar e aplicar algoritmos adequados ao tipo de problema (classificação, regressão, agrupamento).
  \item Avaliar modelos utilizando métricas apropriadas e técnicas de validação.
  \item Documentar e comunicar o processo de modelagem e os resultados obtidos.
\end{itemize}

\textbf{Conteúdos:}
\begin{itemize}
  \item Introdução ao fluxo de trabalho de modelagem preditiva.
  \item Preparação dos dados para modelagem:
    \begin{itemize}
      \item Divisão em conjuntos de treino, validação e teste.
      \item Normalização e padronização de variáveis.
      \item Codificação de variáveis categóricas.
    \end{itemize}
  \item Engenharia de variáveis:
    \begin{itemize}
      \item Criação de novas variáveis a partir de dados existentes.
      \item Seleção de variáveis relevantes (\textit{feature selection}).
      \item Redução de dimensionalidade (PCA).
    \end{itemize}
  \item Treino de modelos:
    \begin{itemize}
      \item Utilização do pacote \texttt{caret} para treino e avaliação.
      \item Fluxo de trabalho com \texttt{tidymodels} e \texttt{recipes}.
      \item Algoritmos comuns: regressão linear, regressão logística, árvores de decisão, \textit{random forest}, k-NN.
    \end{itemize}
  \item Avaliação de modelos:
    \begin{itemize}
      \item Métricas para regressão: RMSE, MAE, $R^2$.
      \item Métricas para classificação: Acurácia, Precisão, Recall, F1-score, AUC.
      \item Validação cruzada e \textit{resampling}.
    \end{itemize}
  \item Comparação e seleção de modelos.
\end{itemize}

\textbf{Atividades Práticas:}
\begin{itemize}
  \item Criar um modelo de regressão para prever preços de imóveis utilizando \texttt{caret}.
  \item Desenvolver um modelo de classificação para prever churn de clientes com \texttt{tidymodels}.
  \item Implementar um \texttt{recipe} para normalizar dados, criar variáveis derivadas e codificar variáveis categóricas.
  \item Comparar o desempenho de pelo menos dois algoritmos diferentes para o mesmo problema.
  \item Documentar o processo de modelagem e apresentar os resultados com métricas e gráficos.
\end{itemize}

\textbf{Recursos e Ferramentas:}
\begin{itemize}
  \item R e RStudio.
  \item Pacotes: \texttt{caret}, \texttt{tidymodels}, \texttt{recipes}, \texttt{ggplot2}, \texttt{dplyr}.
  \item Datasets de apoio: preços de imóveis, churn de clientes, datasets públicos do UCI Machine Learning Repository.
\end{itemize}




%\subsection*{Módulo 4 – Reporting and Documentation (8h)}
%\textbf{Objetivos:}
%\begin{itemize}
%  \item Criar relatórios e dashboards interativos
%  \item Comunicar resultados para públicos técnicos e não técnicos
%\end{itemize}
%
%\textbf{Ferramentas:} \texttt{rmarkdown}, \texttt{flexdashboard}, \texttt{shiny}


\subsection*{Módulo 4 – Reporting and Documentation (8h)}

\textbf{Objetivos:}
\begin{itemize}
  \item Compreender a importância da comunicação clara e estruturada dos resultados em projetos de Data Science.
  \item Criar relatórios técnicos e executivos que transmitam de forma eficaz as descobertas e conclusões.
  \item Desenvolver dashboards interativos para visualização e monitorização de métricas e indicadores.
  \item Adaptar a comunicação para diferentes públicos-alvo (técnico e não técnico).
  \item Documentar o código, processos e decisões para garantir reprodutibilidade e manutenção futura.
\end{itemize}

\textbf{Conteúdos:}
\begin{itemize}
  \item Boas práticas de comunicação de resultados em Data Science.
  \item Relatórios com \texttt{RMarkdown}:
    \begin{itemize}
      \item Estrutura de um documento técnico.
      \item Inclusão de código, tabelas e gráficos.
      \item Exportação para HTML, PDF e Word.
    \end{itemize}
  \item Dashboards interativos:
    \begin{itemize}
      \item Criação com \texttt{flexdashboard}.
      \item Aplicações web com \texttt{shiny}.
      \item Integração de visualizações dinâmicas (\texttt{plotly}, \texttt{leaflet}).
    \end{itemize}
  \item Storytelling com dados:
    \begin{itemize}
      \item Estrutura narrativa para apresentação de insights.
      \item Uso de visualizações para reforçar mensagens-chave.
    \end{itemize}
  \item Documentação de código e processos:
    \begin{itemize}
      \item Comentários claros e consistentes.
      \item Ficheiros \texttt{README} e guias de utilização.
      \item Versionamento com Git/GitHub.
    \end{itemize}
\end{itemize}

\textbf{Atividades Práticas:}
\begin{itemize}
  \item Criar um relatório técnico em \texttt{RMarkdown} com análise exploratória e resultados de um modelo preditivo.
  \item Desenvolver um dashboard interativo com \texttt{flexdashboard} ou \texttt{shiny} para monitorizar indicadores-chave.
  \item Preparar uma apresentação executiva com gráficos e conclusões para um público não técnico.
  \item Documentar todo o processo de análise, incluindo código, decisões e fontes de dados.
\end{itemize}

\textbf{Recursos e Ferramentas:}
\begin{itemize}
  \item R e RStudio.
  \item Pacotes: \texttt{rmarkdown}, \texttt{flexdashboard}, \texttt{shiny}, \texttt{plotly}, \texttt{leaflet}.
  \item Ferramentas de versionamento: Git e GitHub.
  \item Datasets de apoio: conjuntos de dados utilizados nos módulos anteriores.
\end{itemize}



%
%\subsection*{Módulo 5 – Research and Learning (6h)}
%\textbf{Objetivos:}
%\begin{itemize}
%  \item Explorar novas bibliotecas e técnicas emergentes
%  \item Desenvolver autonomia na aprendizagem contínua
%\end{itemize}


\subsection*{Módulo 5 – Research and Learning (6h)}

\textbf{Objetivos:}
\begin{itemize}
  \item Desenvolver a capacidade de pesquisa autónoma e contínua em Data Science, Machine Learning e Inteligência Artificial.
  \item Identificar e avaliar novas bibliotecas, pacotes e técnicas relevantes para projetos em R.
  \item Manter-se atualizado com as tendências e avanços tecnológicos na área.
  \item Integrar novos conhecimentos e ferramentas em projetos práticos.
  \item Fomentar a aprendizagem colaborativa e a partilha de conhecimento dentro da equipa.
\end{itemize}

\textbf{Conteúdos:}
\begin{itemize}
  \item Fontes de informação e atualização:
    \begin{itemize}
      \item Repositórios oficiais (CRAN, Bioconductor).
      \item Comunidades e fóruns (\textit{Stack Overflow}, RStudio Community, R-bloggers).
      \item Publicações científicas e técnicas (arXiv, IEEE, ACM).
    \end{itemize}
  \item Avaliação de novas ferramentas:
    \begin{itemize}
      \item Critérios de seleção (popularidade, manutenção, documentação, compatibilidade).
      \item Testes de desempenho e integração.
    \end{itemize}
  \item Aprendizagem contínua:
    \begin{itemize}
      \item Cursos online (Coursera, edX, DataCamp).
      \item Participação em conferências e \textit{meetups} (useR!, R/Finance, EARL).
      \item Leitura de \textit{white papers} e estudos de caso.
    \end{itemize}
  \item Partilha de conhecimento:
    \begin{itemize}
      \item Apresentações internas.
      \item Publicação de artigos técnicos ou tutoriais.
      \item Contribuição para projetos \textit{open source}.
    \end{itemize}
\end{itemize}

\textbf{Atividades Práticas:}
\begin{itemize}
  \item Pesquisar e apresentar um pacote R recente, explicando a sua utilidade e aplicabilidade.
  \item Implementar um exemplo prático com uma técnica ou biblioteca recém-descoberta.
  \item Criar um breve relatório ou \texttt{RMarkdown} com a avaliação de uma nova ferramenta.
  \item Participar num fórum ou comunidade online, contribuindo com uma resposta ou tutorial.
\end{itemize}

\textbf{Recursos e Ferramentas:}
\begin{itemize}
  \item R e RStudio.
  \item Acesso a internet para pesquisa e participação em comunidades.
  \item Plataformas de cursos online (Coursera, edX, DataCamp).
  \item Repositórios de código (GitHub, GitLab).
\end{itemize}




%\subsection*{Módulo 6 – Ad-hoc Analysis (8h)}
%\textbf{Objetivos:}
%\begin{itemize}
%  \item Responder a perguntas específicas com análise orientada
%  \item Gerar recomendações baseadas em evidências
%\end{itemize}

\subsection*{Módulo 6 – Ad-hoc Analysis (8h)}

\textbf{Objetivos:}
\begin{itemize}
  \item Compreender o conceito e a importância da análise \textit{ad-hoc} para responder a questões específicas de negócio ou investigação.
  \item Desenvolver capacidade de formular hipóteses e estruturar análises rápidas e direcionadas.
  \item Utilizar R para realizar consultas, filtragens e cálculos específicos de forma eficiente.
  \item Produzir resultados claros e acionáveis a partir de análises pontuais.
  \item Comunicar conclusões de forma objetiva e adaptada ao público-alvo.
\end{itemize}

\textbf{Conteúdos:}
\begin{itemize}
  \item Introdução à análise \textit{ad-hoc}:
    \begin{itemize}
      \item Diferença entre análise exploratória e análise \textit{ad-hoc}.
      \item Quando e por que utilizar este tipo de abordagem.
    \end{itemize}
  \item Formulação de perguntas e hipóteses:
    \begin{itemize}
      \item Identificação de necessidades de informação.
      \item Definição de métricas e indicadores relevantes.
    \end{itemize}
  \item Técnicas de análise rápida em R:
    \begin{itemize}
      \item Filtragem e agregação de dados com \texttt{dplyr}.
      \item Criação de resumos estatísticos direcionados.
      \item Visualizações rápidas para suporte à decisão.
    \end{itemize}
  \item Boas práticas na apresentação de resultados:
    \begin{itemize}
      \item Clareza e objetividade.
      \item Uso de visualizações simples e diretas.
      \item Contextualização das conclusões.
    \end{itemize}
\end{itemize}

\textbf{Atividades Práticas:}
\begin{itemize}
  \item Receber uma questão de negócio simulada (ex.: “Quais os 5 produtos mais vendidos no último trimestre por região?”) e responder com análise em R.
  \item Criar um relatório breve com tabelas e gráficos que respondam a uma pergunta específica.
  \item Utilizar \texttt{ggplot2} para criar visualizações rápidas que suportem a resposta.
  \item Apresentar conclusões e recomendações baseadas na análise realizada.
\end{itemize}

\textbf{Recursos e Ferramentas:}
\begin{itemize}
  \item R e RStudio.
  \item Pacotes: \texttt{dplyr}, \texttt{ggplot2}, \texttt{lubridate}.
  \item Datasets de apoio: vendas online, dados de saúde pública, dados meteorológicos.
\end{itemize}



%\subsection*{Módulo 7 – Tooling and Infrastructure (6h)}
%\textbf{Objetivos:}
%\begin{itemize}
%  \item Manter e otimizar o ambiente de trabalho em R
%  \item Integrar R com outras ferramentas e plataformas
%\end{itemize}


\subsection*{Módulo 7 – Tooling and Infrastructure (6h)}

\textbf{Objetivos:}
\begin{itemize}
  \item Compreender a importância das ferramentas e da infraestrutura no suporte a projetos de Data Science.
  \item Configurar e manter ambientes de desenvolvimento reprodutíveis e eficientes em R.
  \item Integrar R com sistemas de controlo de versões e plataformas colaborativas.
  \item Automatizar tarefas e fluxos de trabalho para aumentar a produtividade.
  \item Garantir a segurança, organização e escalabilidade dos projetos.
\end{itemize}

\textbf{Conteúdos:}
\begin{itemize}
  \item Organização de projetos:
    \begin{itemize}
      \item Estrutura de pastas e convenções de nomenclatura.
      \item Utilização do pacote \texttt{here} para caminhos relativos.
    \end{itemize}
  \item Gestão de dependências e ambientes:
    \begin{itemize}
      \item Criação de ambientes isolados com \texttt{renv}.
      \item Documentação de pacotes utilizados.
    \end{itemize}
  \item Controlo de versões:
    \begin{itemize}
      \item Introdução ao Git e GitHub.
      \item Fluxos de trabalho colaborativos (\textit{branching}, \textit{pull requests}).
    \end{itemize}
  \item Automatização de tarefas:
    \begin{itemize}
      \item Pipelines de análise com \texttt{targets} ou \texttt{drake}.
      \item Agendamento de scripts com \texttt{cronR} ou tarefas agendadas no sistema operativo.
    \end{itemize}
  \item Integração e interoperabilidade:
    \begin{itemize}
      \item Comunicação entre R e outras linguagens (Python via \texttt{reticulate}).
      \item Conexão com APIs e bases de dados.
    \end{itemize}
  \item Boas práticas de segurança e backup.
\end{itemize}

\textbf{Atividades Práticas:}
\begin{itemize}
  \item Criar a estrutura de um projeto em R com \texttt{here} e \texttt{renv}.
  \item Configurar um repositório GitHub e realizar operações básicas (commit, push, pull).
  \item Implementar um pipeline de análise com \texttt{targets} para automatizar a execução de scripts.
  \item Agendar a execução automática de um script de recolha e limpeza de dados.
  \item Documentar o ambiente e as dependências do projeto para partilha com outros utilizadores.
\end{itemize}

\textbf{Recursos e Ferramentas:}
\begin{itemize}
  \item R e RStudio.
  \item Pacotes: \texttt{here}, \texttt{renv}, \texttt{targets}, \texttt{drake}, \texttt{reticulate}, \texttt{cronR}.
  \item Ferramentas de controlo de versões: Git e GitHub.
  \item Plataformas colaborativas: GitHub, GitLab.
\end{itemize}

\subsection*{Módulo 8 – Deployment e Manutenção de Modelos (6h)}

\textbf{Objetivos:}
\begin{itemize}
  \item Compreender o processo de disponibilização (\textit{deployment}) de modelos de Machine Learning em ambientes de produção.
  \item Implementar soluções para servir modelos desenvolvidos em R de forma segura e escalável.
  \item Monitorizar o desempenho de modelos após a implementação.
  \item Aplicar técnicas de manutenção e atualização de modelos para garantir a sua relevância e eficácia ao longo do tempo.
  \item Documentar e comunicar o processo de \textit{deployment} e manutenção a equipas técnicas e não técnicas.
\end{itemize}

\textbf{Conteúdos:}
\begin{itemize}
  \item Conceitos de \textit{deployment}:
    \begin{itemize}
      \item Diferença entre ambiente de desenvolvimento, teste e produção.
      \item Ciclo de vida de um modelo em produção.
    \end{itemize}
  \item Ferramentas e abordagens para \textit{deployment} em R:
    \begin{itemize}
      \item APIs com \texttt{plumber}.
      \item Aplicações web com \texttt{shiny}.
      \item Integração com serviços externos (Docker, cloud computing).
    \end{itemize}
  \item Monitorização de modelos:
    \begin{itemize}
      \item Métricas de desempenho em produção.
      \item Deteção de \textit{data drift} e \textit{concept drift}.
    \end{itemize}
  \item Manutenção e atualização:
    \begin{itemize}
      \item Re-treino periódico.
      \item Gestão de versões de modelos.
      \item Automação de pipelines de atualização.
    \end{itemize}
  \item Boas práticas de segurança e escalabilidade.
\end{itemize}

\textbf{Atividades Práticas:}
\begin{itemize}
  \item Criar uma API simples com \texttt{plumber} para servir previsões de um modelo treinado.
  \item Desenvolver uma aplicação \texttt{shiny} para visualização de resultados e interação com o modelo.
  \item Configurar um processo de monitorização de métricas de desempenho.
  \item Simular um cenário de \textit{data drift} e aplicar um re-treino do modelo.
  \item Documentar o processo de \textit{deployment} e manutenção num relatório técnico.
\end{itemize}

\textbf{Recursos e Ferramentas:}
\begin{itemize}
  \item R e RStudio.
  \item Pacotes: \texttt{plumber}, \texttt{shiny}, \texttt{caret} ou \texttt{tidymodels}, \texttt{ggplot2}.
  \item Ferramentas externas: Docker, GitHub Actions, serviços de cloud (AWS, Azure, GCP).
  \item Datasets de apoio: modelos desenvolvidos nos módulos anteriores.
\end{itemize}


\subsection*{Módulo 9 – Avaliação (4h)}

\textbf{Objetivos:}
\begin{itemize}
  \item Definir critérios claros e objetivos para avaliar o desempenho dos formandos.
  \item Aplicar instrumentos de avaliação diagnóstica, formativa e sumativa.
  \item Garantir que a avaliação reflete as competências técnicas e comportamentais desenvolvidas.
  \item Fornecer feedback construtivo para promover a melhoria contínua.
  \item Documentar os resultados da avaliação de forma transparente e organizada.
\end{itemize}

\textbf{Conteúdos:}
\begin{itemize}
  \item Tipos de avaliação:
    \begin{itemize}
      \item Diagnóstica – levantamento de conhecimentos prévios.
      \item Formativa – acompanhamento contínuo do progresso.
      \item Sumativa – medição final das competências adquiridas.
    \end{itemize}
  \item Critérios de avaliação:
    \begin{itemize}
      \item Domínio técnico (execução correta de tarefas, uso adequado de ferramentas).
      \item Resolução de problemas e pensamento crítico.
      \item Clareza e qualidade da comunicação de resultados.
      \item Participação, colaboração e gestão do tempo.
    \end{itemize}
  \item Instrumentos de avaliação:
    \begin{itemize}
      \item Fichas de trabalho e exercícios práticos.
      \item Projetos individuais e em grupo.
      \item Testes teóricos e práticos.
      \item Apresentações orais e relatórios escritos.
    \end{itemize}
  \item Feedback:
    \begin{itemize}
      \item Técnicas de feedback construtivo.
      \item Autoavaliação e avaliação por pares.
    \end{itemize}
\end{itemize}

\textbf{Atividades Práticas:}
\begin{itemize}
  \item Aplicar um questionário diagnóstico no início do curso.
  \item Avaliar uma ficha prática com base numa grelha de critérios predefinidos.
  \item Realizar uma sessão de feedback individual com cada formando.
  \item Conduzir a avaliação final através de um projeto integrador.
\end{itemize}

\textbf{Recursos e Ferramentas:}
\begin{itemize}
  \item Grelhas de avaliação e rubricas.
  \item Plataformas de gestão de aprendizagem (LMS) para submissão e feedback.
  \item Ferramentas de colaboração online (Google Workspace, Microsoft 365).
\end{itemize}

\textbf{Ponderação Sugerida:}
\begin{center}
\begin{tabular}{l c}
\toprule
\textbf{Tipo de Avaliação} & \textbf{Peso} \\
\midrule
Diagnóstica & -- \\
Formativa (fichas, participação, exercícios) & 30\% \\
Sumativa (projeto final, teste teórico-prático) & 70\% \\
\bottomrule
\end{tabular}
\end{center}






%\section*{Avaliação}
%
%\begin{tabular}{ll}
%\toprule
%\textbf{Tipo} & \textbf{Peso} \\
%\midrule
%Contínua (participação, exercícios práticos) & 30\% \\
%Projetos (relatórios, dashboards, modelos) & 50\% \\
%Final (apresentação oral + documentação técnica) & 20\% \\
%\bottomrule
%\end{tabular}

\subsection*{Grelha de Avaliação – Projeto Final}

\textbf{Objetivo:} Avaliar o desempenho dos formandos na execução do projeto final, considerando competências técnicas, metodológicas e comunicacionais.

\begin{center}
\renewcommand{\arraystretch}{1.5}
\begin{tabular}{p{4cm} p{7cm} c}
\toprule
\textbf{Critério} & \textbf{Descritores de Desempenho} & \textbf{Peso} \\
\midrule

\textbf{1. Planeamento e Organização} &
\begin{itemize}[leftmargin=*]
  \item Excelente: Objetivos claros, plano de trabalho estruturado, gestão de tempo eficaz.
  \item Bom: Objetivos definidos, plano de trabalho adequado, gestão de tempo satisfatória.
  \item Insuficiente: Objetivos pouco claros, ausência de planeamento, má gestão de tempo.
\end{itemize}
& 10\% \\

\textbf{2. Preparação e Qualidade dos Dados} &
\begin{itemize}[leftmargin=*]
  \item Excelente: Dados limpos, consistentes e bem documentados; aplicação correta de técnicas de preparação.
  \item Bom: Dados preparados com pequenas inconsistências; aplicação adequada de técnicas.
  \item Insuficiente: Dados com erros significativos; preparação incompleta ou incorreta.
\end{itemize}
& 15\% \\

\textbf{3. Análise Exploratória (EDA)} &
\begin{itemize}[leftmargin=*]
  \item Excelente: Análise completa, identificação clara de padrões, outliers e correlações; visualizações eficazes.
  \item Bom: Análise adequada, identificação de alguns padrões e relações; visualizações satisfatórias.
  \item Insuficiente: Análise superficial, ausência de insights relevantes; visualizações pouco claras.
\end{itemize}
& 15\% \\

\textbf{4. Desenvolvimento do Modelo} &
\begin{itemize}[leftmargin=*]
  \item Excelente: Seleção apropriada do algoritmo, engenharia de variáveis eficaz, treino e validação corretos.
  \item Bom: Algoritmo adequado, engenharia de variáveis satisfatória, treino e validação aceitáveis.
  \item Insuficiente: Algoritmo inadequado, engenharia de variáveis insuficiente, treino/validação incorretos.
\end{itemize}
& 20\% \\


\textbf{5. Avaliação e Interpretação dos Resultados} &
\begin{itemize}[leftmargin=*]
  \item Excelente: Métricas calculadas corretamente, interpretação clara e fundamentada, comparação entre modelos.
  \item Bom: Métricas corretas, interpretação adequada, comparação limitada.
  \item Insuficiente: Métricas incorretas ou incompletas, interpretação fraca ou ausente.
\end{itemize}
& 15\% \\

\textbf{6. Comunicação e Apresentação} &
\begin{itemize}[leftmargin=*]
  \item Excelente: Relatório e/ou apresentação claros, bem estruturados, com visualizações de qualidade e linguagem adequada ao público.
  \item Bom: Relatório/apresentação adequados, estrutura satisfatória, visualizações compreensíveis.
  \item Insuficiente: Relatório/apresentação confusos, estrutura deficiente, visualizações pouco claras.
\end{itemize}
& 15\% \\

\textbf{7. Documentação e Reprodutibilidade} &
\begin{itemize}[leftmargin=*]
  \item Excelente: Código bem comentado, documentação completa, projeto totalmente reprodutível.
  \item Bom: Código comentado, documentação suficiente, projeto parcialmente reprodutível.
  \item Insuficiente: Código sem comentários, documentação insuficiente, projeto não reprodutível.
\end{itemize}
& 10\% \\

\midrule
\textbf{Total} & & \textbf{100\%} \\
\bottomrule
\end{tabular}
\end{center}

\textbf{Observações do Avaliador:} \\
\rule{\textwidth}{0.4pt} \\
\vspace{1cm}
\rule{\textwidth}{0.4pt}


\subsection*{Grelha de Avaliação – Projeto Final (Escala 1 a 5)}

\textbf{Objetivo:} Avaliar o desempenho dos formandos na execução do projeto final, atribuindo pontuações objetivas com base em critérios definidos.

\begin{center}
\renewcommand{\arraystretch}{1.5}
\begin{tabular}{p{4cm} p{6.5cm} c c}
\toprule
\textbf{Critério} & \textbf{Descritores de Desempenho} & \textbf{Peso} & \textbf{Pontuação (1-5)} \\
\midrule

\textbf{1. Planeamento e Organização} &
1 = Sem planeamento; 3 = Planeamento básico; 5 = Planeamento completo, objetivos claros e gestão de tempo eficaz. & 10\% & \\

\textbf{2. Preparação e Qualidade dos Dados} &
1 = Dados com erros graves; 3 = Dados preparados com pequenas falhas; 5 = Dados limpos, consistentes e bem documentados. & 15\% & \\

\textbf{3. Análise Exploratória (EDA)} &
1 = Análise superficial; 3 = Análise adequada; 5 = Análise completa com insights relevantes e visualizações eficazes. & 15\% & \\

\textbf{4. Desenvolvimento do Modelo} &
1 = Algoritmo inadequado; 3 = Algoritmo adequado mas otimização limitada; 5 = Algoritmo bem escolhido, otimizado e validado. & 20\% & \\

\textbf{5. Avaliação e Interpretação dos Resultados} &
1 = Métricas incorretas ou mal interpretadas; 3 = Métricas corretas mas interpretação limitada; 5 = Métricas corretas, interpretação clara e fundamentada. & 15\% & \\

\textbf{6. Comunicação e Apresentação} &
1 = Apresentação confusa; 3 = Apresentação adequada; 5 = Apresentação clara, estruturada e adaptada ao público. & 15\% & \\

\textbf{7. Documentação e Reprodutibilidade} &
1 = Código sem comentários e não reprodutível; 3 = Documentação parcial; 5 = Código bem comentado, documentação completa e projeto reprodutível. & 10\% & \\

\midrule
\textbf{Total} & & \textbf{100\%} & \\
\bottomrule
\end{tabular}
\end{center}

\textbf{Escala de Pontuação:}
\begin{itemize}
  \item 1 – Muito Insuficiente
  \item 2 – Insuficiente
  \item 3 – Satisfatório
  \item 4 – Bom
  \item 5 – Excelente
\end{itemize}

\textbf{Cálculo da Nota Final:}  


\[
\text{Nota Final} = \frac{\sum (\text{Pontuação} \times \text{Peso})}{\text{Soma dos Pesos}}
\]



\textbf{Observações do Avaliador:} \\
\rule{\textwidth}{0.4pt} \\
\vspace{1cm}
\rule{\textwidth}{0.4pt}



\section*{Quiz de Revisão}

\begin{enumerate}
  \item Qual pacote é ideal para limpeza de dados em R?\\
  \textbf{a)} ggplot2 \quad \textbf{b)} janitor \quad \textbf{c)} shiny \quad \textbf{d)} lubridate\\
  \textbf{Resposta correta: b}

  \item Qual função é usada para dividir dados em treino e teste?\\
  \textbf{a)} split() \quad \textbf{b)} sample() \quad \textbf{c)} initial\_split() \quad \textbf{d)} divide\_data()\\
  \textbf{Resposta correta: c}

  \item Qual métrica avalia modelos de regressão?\\
  \textbf{a)} AUC \quad \textbf{b)} RMSE \quad \textbf{c)} Accuracy \quad \textbf{d)} Recall\\
  \textbf{Resposta correta: b}

  \item Qual ferramenta permite criar dashboards interativos em R?\\
  \textbf{a)} rmarkdown \quad \textbf{b)} shiny \quad \textbf{c)} ggplot2 \quad \textbf{d)} tidyr\\
  \textbf{Resposta correta: b}

  \item Qual pacote ajuda a gerir dependências e ambientes em R?\\
  \textbf{a)} dplyr \quad \textbf{b)} renv \quad \textbf{c)} caret \quad \textbf{d)} readr\\
  \textbf{Resposta correta: b}
\end{enumerate}

\end{document}
