\begin{lstlisting}[language=R]
library(tidyverse)
library(plotly)
library(corrplot)

# 1. Import dataset
dados <- read_csv("vendas.csv")

# 2. Descriptive statistics
summary(dados)
dados %>% summarise(
  media_preco = mean(preco, na.rm = TRUE),
  mediana_preco = median(preco, na.rm = TRUE),
  desvio_padrao_preco = sd(preco, na.rm = TRUE)
)

# 3. Basic visualizations
ggplot(dados, aes(x = preco)) + geom_histogram(binwidth = 5)
ggplot(dados, aes(x = categoria, y = preco)) + geom_boxplot()

# 4. Interactive visualization
grafico_interativo <- ggplot(dados, aes(x = preco, y = quantidade, color = categoria)) +
  geom_point()
ggplotly(grafico_interativo)

# 5. Correlation matrix
matriz <- cor(dados %>% select_if(is.numeric), use = "complete.obs")
corrplot(matriz, method = "color", type = "upper", tl.col = "black")

# 6. Outlier identification (IQR)
Q1 <- quantile(dados$preco, 0.25, na.rm = TRUE)
Q3 <- quantile(dados$preco, 0.75, na.rm = TRUE)
IQR <- Q3 - Q1
outliers <- dados %>% filter(preco < (Q1 - 1.5*IQR) | preco > (Q3 + 1.5*IQR))
\end{lstlisting}

**Notes for the Instructor:**
* The \texttt{vendas.csv} dataset can be real or synthetic, but it should contain both numerical and categorical variables.
* Encourage students to experiment with different chart types and parameters.
* Discuss how EDA findings can influence subsequent modeling.