\section{\textcolor{sectionred}{Module 6 – Ad-hoc Analysis (8h)}}

\subsection{\textcolor{subsectionblue}{Objectives}}
By the end of this module, students should be able to:
\begin{itemize}
  \item Understand the role of **ad-hoc** analyses in quickly responding to specific business or research questions.
  \item Formulate hypotheses and structure targeted analyses to answer concrete questions.
  \item Select and apply statistical and visualization methods suitable for the problem.
  \item Communicate results clearly and objectively, focusing on the most relevant conclusions.
\end{itemize}

\subsection{\textcolor{subsectionblue}{Developed Contents}}
\begin{itemize}
  \item \textbf{Definition and context of ad-hoc analyses} — difference between planned and ad-hoc analyses; when and why to use them.
  \item \textbf{Formulation of hypotheses and business questions} — transforming vague questions into testable hypotheses.
  \item \textbf{Selection of methods and tools} — choosing appropriate statistical techniques, filters, and visualizations.
  \item \textbf{Rapid execution and validation of results} — ensuring that speed does not compromise quality.
  \item \textbf{Communication and delivery} — delivery formats: brief report, email, short presentation.
\end{itemize}

\subsection{\textcolor{subsectionblue}{Detailed Practical Activities}}
\begin{itemize}
  \item Receive a simulated business question and develop an ad-hoc analysis to answer it.
  \item Create simple and effective visualizations to communicate results.
  \item Apply a statistical test suitable for the presented problem.
  \item Document the analysis in a concise and clear format.
\end{itemize}

\subsection{\textcolor{subsectionblue}{Resources and Tools}}
\begin{itemize}
  \item \textbf{Software:} R and RStudio.
  \item \textbf{Packages:} \texttt{tidyverse}, \texttt{lubridate}, \texttt{ggplot2}, \texttt{broom}.
  \item \textbf{Data sources:} internal or public datasets simulating real-world scenarios.
\end{itemize}

\subsection{\textcolor{subsectionblue}{Case Study – Ad-hoc Analysis in R}}
\textbf{Objective:} Demonstrate how to quickly answer a specific question with available data.

\begin{lstlisting}[language=R]
library(tidyverse)
library(lubridate)
library(broom)

# Scenario: The manager wants to know if the average sales of the last month
# were significantly different from the previous month.

# 1. Import data
dados <- read_csv("vendas.csv") %>%
  mutate(data = dmy(data))

# 2. Filter last two months
ultimo_mes <- max(month(dados$data))
dados_filtrados <- dados %>%
  filter(month(data) %in% c(ultimo_mes, ultimo_mes - 1))

# 3. Descriptive statistics
dados_filtrados %>%
  group_by(mes = month(data)) %>%
  summarise(media_vendas = mean(vendas, na.rm = TRUE),
            sd_vendas = sd(vendas, na.rm = TRUE))

# 4. Statistical test (t-test)
resultado_t <- t.test(vendas ~ month(data), data = dados_filtrados)
tidy(resultado_t)

# 5. Quick visualization
ggplot(dados_filtrados, aes(x = factor(month(data)), y = vendas)) +
  geom_boxplot(fill = "skyblue") +
  labs(x = "Month", y = "Sales", title = "Sales Comparison Between Months")

# 6. Conclusion
# If p-value < 0.05, there is a statistically significant difference.
\end{lstlisting}

\textbf{Notes for the Instructor:}
\begin{itemize}
  \item This example shows how to quickly structure an analysis from data import to conclusion.
  \item Encourage students to adapt the code to other types of ad-hoc questions.
  \item Discuss the importance of communicating only the essentials in urgent analyses.
\end{itemize}