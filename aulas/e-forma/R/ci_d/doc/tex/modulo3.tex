\section{\textcolor{sectionred}{Módulo 3 – Model Development Support (12h)}}

\subsection{\textcolor{subsectionblue}{Objetivos}}
Ao final deste módulo, os formandos deverão ser capazes de:
\begin{itemize}
  \item Compreender o papel do desenvolvimento de modelos no ciclo de vida de um projeto de Ciência de Dados.  
  \textit{Exemplo prático:} Uma empresa de logística desenvolve um modelo para prever atrasos nas entregas com base em dados históricos e meteorológicos.
  \item Apoiar na criação de modelos preditivos supervisionados e não supervisionados.  
  \item Realizar engenharia de variáveis (\textit{feature engineering}) para melhorar a performance dos modelos.  
  \item Selecionar e aplicar algoritmos adequados ao tipo de problema.  
  \item Avaliar modelos utilizando métricas apropriadas e técnicas de validação.  
  \item Documentar e comunicar o processo de modelagem e os resultados obtidos.
\end{itemize}

\subsection{\textcolor{subsectionblue}{Conteúdos Desenvolvidos}}
\begin{itemize}
  \item \textbf{Introdução ao fluxo de trabalho de modelagem preditiva} — preparação de dados, treino, validação, teste e implementação.
  \item \textbf{Preparação dos dados:} divisão treino/validação/teste; normalização e padronização; codificação de variáveis categóricas.
  \item \textbf{Engenharia de variáveis:} criação de novas variáveis; seleção de variáveis relevantes; redução de dimensionalidade (PCA).
  \item \textbf{Treino de modelos:} uso de \texttt{caret}, \texttt{tidymodels} e \texttt{recipes}; algoritmos comuns (regressão linear/logística, árvores, random forest, k-NN).
  \item \textbf{Avaliação de modelos:} métricas para regressão (RMSE, MAE, $R^2$) e classificação (Acurácia, Precisão, Recall, F1-score, AUC); validação cruzada.
  \item \textbf{Comparação e seleção de modelos}.
\end{itemize}

\subsection{\textcolor{subsectionblue}{Atividades Práticas Detalhadas}}
\begin{itemize}
  \item Criar um modelo de regressão para prever preços de imóveis utilizando \texttt{caret}.
  \item Desenvolver um modelo de classificação para prever churn de clientes com \texttt{tidymodels}.
  \item Implementar um \texttt{recipe} para normalizar dados, criar variáveis derivadas e codificar variáveis categóricas.
  \item Comparar o desempenho de pelo menos dois algoritmos diferentes para o mesmo problema.
  \item Documentar o processo de modelagem e apresentar os resultados com métricas e gráficos.
\end{itemize}

\subsection{\textcolor{subsectionblue}{Recursos e Ferramentas}}
\begin{itemize}
  \item \textbf{Software:} R e RStudio.
  \item \textbf{Pacotes:} \texttt{caret}, \texttt{tidymodels}, \texttt{recipes}, \texttt{ggplot2}, \texttt{dplyr}.
  \item \textbf{Datasets:} preços de imóveis, churn de clientes, datasets públicos do UCI Machine Learning Repository.
\end{itemize}

\subsection{\textcolor{subsectionblue}{Estudo de Caso – Desenvolvimento de Modelo em R}}
\textbf{Objetivo:} Demonstrar todas as etapas do desenvolvimento de um modelo preditivo.

\begin{lstlisting}[language=R]
library(tidyverse)
library(caret)
library(tidymodels)

# 1. Importar dataset
dados <- read_csv("precos_imoveis.csv")

# 2. Divisão treino/teste
set.seed(123)
divisao <- initial_split(dados, prop = 0.7)
treino <- training(divisao)
teste <- testing(divisao)

# 3. Recipe de pré-processamento
receita <- recipe(preco ~ ., data = treino) %>%
  step_normalize(all_numeric_predictors()) %>%
  step_dummy(all_nominal_predictors())

# 4. Modelo regressão linear
modelo_rl <- linear_reg() %>% set_engine("lm")

# 5. Workflow
workflow_rl <- workflow() %>%
  add_recipe(receita) %>%
  add_model(modelo_rl)

# 6. Treinar
modelo_treinado <- fit(workflow_rl, data = treino)

# 7. Avaliar
predicoes <- predict(modelo_treinado, teste) %>%
  bind_cols(teste)
metrics(predicoes, truth = preco, estimate = .pred)

# 8. Comparar com Random Forest
modelo_rf <- rand_forest() %>%
  set_engine("ranger") %>%
  set_mode("regression")
workflow_rf <- workflow() %>%
  add_recipe(receita) %>%
  add_model(modelo_rf)
modelo_rf_treinado <- fit(workflow_rf, data = treino)
predicoes_rf <- predict(modelo_rf_treinado, teste) %>%
  bind_cols(teste)
metrics(predicoes_rf, truth = preco, estimate = .pred)
\end{lstlisting}

\textbf{Notas para o Formador:}
\begin{itemize}
  \item Este estudo de caso cobre preparação de dados, treino, avaliação e comparação de algoritmos.
  \item O dataset pode ser obtido de fontes abertas ou gerado sinteticamente.
  \item Incentivar a experimentação com diferentes algoritmos e hiperparâmetros.
  \item Discutir a escolha de métricas adequadas ao problema.
\end{itemize}
