\section{\textcolor{sectionred}{Module 5 – Research and Learning (6h)}}

\subsection{\textcolor{subsectionblue}{Objectives}}
By the end of this module, students should be able to:
\begin{itemize}
  \item Develop continuous research skills to stay updated in the fields of Data Science, Machine Learning, and Artificial Intelligence.  
  \item Critically evaluate new libraries, techniques, and methodologies before adopting them.  
  \item Integrate new tools and approaches into existing workflows.  
  \item Share knowledge with the community and the work team.
\end{itemize}

\subsection{\textcolor{subsectionblue}{Developed Contents}}
\begin{itemize}
  \item \textbf{Information and update sources:} package repositories (CRAN, Bioconductor), blogs and communities (\texttt{R-bloggers}, Stack Overflow, Posit Community), scientific publications (arXiv, IEEE, Nature Machine Intelligence).
  \item \textbf{Evaluation of new tools:} evaluation criteria (performance, documentation, community, maintenance), integration tests in pilot projects.
  \item \textbf{Workflow integration:} adapting code and pipelines, automating repetitive tasks.
  \item \textbf{Knowledge sharing:} internal and external documentation, contributing to open-source projects.
\end{itemize}

\subsection{\textcolor{subsectionblue}{Detailed Practical Activities}}
\begin{itemize}
  \item Research and present a recent R package, explaining its functionalities and use cases.
  \item Implement a practical example with a newly discovered technique or package.
  \item Create a comparative document between two approaches for the same problem.
  \item Publish a small tutorial or code example in a GitHub repository.
\end{itemize}

\subsection{\textcolor{subsectionblue}{Resources and Tools}}
\begin{itemize}
  \textbf{Software:} R and RStudio.
  \textbf{Packages:} variable according to research (e.g., \texttt{tidyverse}, \texttt{data.table}, \texttt{lightgbm}).
  \textbf{Data sources:} CRAN, GitHub, arXiv, public APIs.
\end{itemize}

\subsection{\textcolor{subsectionblue}{Case Study – Research and Integration of a New Package}}
\textbf{Objective:} Demonstrate how to identify, test, and integrate a new R package into a workflow.

\begin{lstlisting}[language=R]
# 1. Package research on CRAN
# Example: 'data.table' package for efficient data manipulation

# 2. Installation and loading
install.packages("data.table")
library(data.table)

# 3. Performance comparison with dplyr
library(tidyverse)
dados <- as.data.frame(matrix(runif(1e6), ncol = 10))

# Using dplyr
system.time({
  df_dplyr <- as_tibble(dados) %>%
    summarise(across(everything(), mean))
})

# Using data.table
system.time({
  dt <- as.data.table(dados)
  dt_means <- dt[, lapply(.SD, mean)]
})

# 4. Workflow integration
# Replace slow operations with data.table equivalents
# Document changes and performance gains

# 5. Sharing results
# Create a README with instructions and benchmarks
\end{lstlisting}

\textbf{Notes for the Instructor:}
\begin{itemize}
  \item Encourage students to choose packages relevant to their interests or work area.
  \item Discuss objective criteria for adopting new tools.
  \item Show how to document and communicate test results to the team.
\end{itemize}