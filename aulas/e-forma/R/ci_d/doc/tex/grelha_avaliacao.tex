% Options for packages loaded elsewhere
\PassOptionsToPackage{unicode}{hyperref}
\PassOptionsToPackage{hyphens}{url}
%
\documentclass[
]{article}
\usepackage{amsmath,amssymb}
\usepackage{iftex}
\ifPDFTeX
  \usepackage[T1]{fontenc}
  \usepackage[utf8]{inputenc}
  \usepackage{textcomp} % provide euro and other symbols
\else % if luatex or xetex
  \usepackage{unicode-math} % this also loads fontspec
  \defaultfontfeatures{Scale=MatchLowercase}
  \defaultfontfeatures[\rmfamily]{Ligatures=TeX,Scale=1}
\fi
\usepackage{lmodern}
\ifPDFTeX\else
  % xetex/luatex font selection
\fi
% Use upquote if available, for straight quotes in verbatim environments
\IfFileExists{upquote.sty}{\usepackage{upquote}}{}
\IfFileExists{microtype.sty}{% use microtype if available
  \usepackage[]{microtype}
  \UseMicrotypeSet[protrusion]{basicmath} % disable protrusion for tt fonts
}{}
\makeatletter
\@ifundefined{KOMAClassName}{% if non-KOMA class
  \IfFileExists{parskip.sty}{%
    \usepackage{parskip}
  }{% else
    \setlength{\parindent}{0pt}
    \setlength{\parskip}{6pt plus 2pt minus 1pt}}
}{% if KOMA class
  \KOMAoptions{parskip=half}}
\makeatother
\usepackage{xcolor}
\usepackage[margin=1in]{geometry}
\usepackage{color}
\usepackage{fancyvrb}
\newcommand{\VerbBar}{|}
\newcommand{\VERB}{\Verb[commandchars=\\\{\}]}
\DefineVerbatimEnvironment{Highlighting}{Verbatim}{commandchars=\\\{\}}
% Add ',fontsize=\small' for more characters per line
\usepackage{framed}
\definecolor{shadecolor}{RGB}{248,248,248}
\newenvironment{Shaded}{\begin{snugshade}}{\end{snugshade}}
\newcommand{\AlertTok}[1]{\textcolor[rgb]{0.94,0.16,0.16}{#1}}
\newcommand{\AnnotationTok}[1]{\textcolor[rgb]{0.56,0.35,0.01}{\textbf{\textit{#1}}}}
\newcommand{\AttributeTok}[1]{\textcolor[rgb]{0.13,0.29,0.53}{#1}}
\newcommand{\BaseNTok}[1]{\textcolor[rgb]{0.00,0.00,0.81}{#1}}
\newcommand{\BuiltInTok}[1]{#1}
\newcommand{\CharTok}[1]{\textcolor[rgb]{0.31,0.60,0.02}{#1}}
\newcommand{\CommentTok}[1]{\textcolor[rgb]{0.56,0.35,0.01}{\textit{#1}}}
\newcommand{\CommentVarTok}[1]{\textcolor[rgb]{0.56,0.35,0.01}{\textbf{\textit{#1}}}}
\newcommand{\ConstantTok}[1]{\textcolor[rgb]{0.56,0.35,0.01}{#1}}
\newcommand{\ControlFlowTok}[1]{\textcolor[rgb]{0.13,0.29,0.53}{\textbf{#1}}}
\newcommand{\DataTypeTok}[1]{\textcolor[rgb]{0.13,0.29,0.53}{#1}}
\newcommand{\DecValTok}[1]{\textcolor[rgb]{0.00,0.00,0.81}{#1}}
\newcommand{\DocumentationTok}[1]{\textcolor[rgb]{0.56,0.35,0.01}{\textbf{\textit{#1}}}}
\newcommand{\ErrorTok}[1]{\textcolor[rgb]{0.64,0.00,0.00}{\textbf{#1}}}
\newcommand{\ExtensionTok}[1]{#1}
\newcommand{\FloatTok}[1]{\textcolor[rgb]{0.00,0.00,0.81}{#1}}
\newcommand{\FunctionTok}[1]{\textcolor[rgb]{0.13,0.29,0.53}{\textbf{#1}}}
\newcommand{\ImportTok}[1]{#1}
\newcommand{\InformationTok}[1]{\textcolor[rgb]{0.56,0.35,0.01}{\textbf{\textit{#1}}}}
\newcommand{\KeywordTok}[1]{\textcolor[rgb]{0.13,0.29,0.53}{\textbf{#1}}}
\newcommand{\NormalTok}[1]{#1}
\newcommand{\OperatorTok}[1]{\textcolor[rgb]{0.81,0.36,0.00}{\textbf{#1}}}
\newcommand{\OtherTok}[1]{\textcolor[rgb]{0.56,0.35,0.01}{#1}}
\newcommand{\PreprocessorTok}[1]{\textcolor[rgb]{0.56,0.35,0.01}{\textit{#1}}}
\newcommand{\RegionMarkerTok}[1]{#1}
\newcommand{\SpecialCharTok}[1]{\textcolor[rgb]{0.81,0.36,0.00}{\textbf{#1}}}
\newcommand{\SpecialStringTok}[1]{\textcolor[rgb]{0.31,0.60,0.02}{#1}}
\newcommand{\StringTok}[1]{\textcolor[rgb]{0.31,0.60,0.02}{#1}}
\newcommand{\VariableTok}[1]{\textcolor[rgb]{0.00,0.00,0.00}{#1}}
\newcommand{\VerbatimStringTok}[1]{\textcolor[rgb]{0.31,0.60,0.02}{#1}}
\newcommand{\WarningTok}[1]{\textcolor[rgb]{0.56,0.35,0.01}{\textbf{\textit{#1}}}}
\usepackage{longtable,booktabs,array}
\usepackage{calc} % for calculating minipage widths
% Correct order of tables after \paragraph or \subparagraph
\usepackage{etoolbox}
\makeatletter
\patchcmd\longtable{\par}{\if@noskipsec\mbox{}\fi\par}{}{}
\makeatother
% Allow footnotes in longtable head/foot
\IfFileExists{footnotehyper.sty}{\usepackage{footnotehyper}}{\usepackage{footnote}}
\makesavenoteenv{longtable}
\usepackage{graphicx}
\makeatletter
\newsavebox\pandoc@box
\newcommand*\pandocbounded[1]{% scales image to fit in text height/width
  \sbox\pandoc@box{#1}%
  \Gscale@div\@tempa{\textheight}{\dimexpr\ht\pandoc@box+\dp\pandoc@box\relax}%
  \Gscale@div\@tempb{\linewidth}{\wd\pandoc@box}%
  \ifdim\@tempb\p@<\@tempa\p@\let\@tempa\@tempb\fi% select the smaller of both
  \ifdim\@tempa\p@<\p@\scalebox{\@tempa}{\usebox\pandoc@box}%
  \else\usebox{\pandoc@box}%
  \fi%
}
% Set default figure placement to htbp
\def\fps@figure{htbp}
\makeatother
\setlength{\emergencystretch}{3em} % prevent overfull lines
\providecommand{\tightlist}{%
  \setlength{\itemsep}{0pt}\setlength{\parskip}{0pt}}
\setcounter{secnumdepth}{-\maxdimen} % remove section numbering
\usepackage{bookmark}
\IfFileExists{xurl.sty}{\usepackage{xurl}}{} % add URL line breaks if available
\urlstyle{same}
\hypersetup{
  pdftitle={Grelha de Avaliação -- Projeto Final},
  pdfauthor={Formador: JOÃO},
  hidelinks,
  pdfcreator={LaTeX via pandoc}}

\title{Grelha de Avaliação -- Projeto Final}
\author{Formador: JOÃO}
\date{}

\begin{document}
\maketitle

Objetivo Avaliar o desempenho dos formandos na execução do projeto
final, atribuindo pontuações objetivas com base em critérios definidos.

Escala de Pontuação 1 -- Muito Insuficiente

2 -- Insuficiente

3 -- Satisfatório

4 -- Bom

5 -- Excelente

Critérios e Pesos

\begin{verbatim}
##                                   Critério Peso Pontuacao Resultado
## 1                Planeamento e Organização 0.10         5      0.50
## 2         Preparação e Qualidade dos Dados 0.15         4      0.60
## 3               Análise Exploratória (EDA) 0.15         4      0.60
## 4                Desenvolvimento do Modelo 0.20         5      1.00
## 5 Avaliação e Interpretação dos Resultados 0.15         4      0.60
## 6               Comunicação e Apresentação 0.15         5      0.75
## 7         Documentação e Reprodutibilidade 0.10         5      0.50
\end{verbatim}

\subsection{Tabela de Avaliação}\label{tabela-de-avaliauxe7uxe3o}

\begin{Shaded}
\begin{Highlighting}[]
\NormalTok{knitr}\SpecialCharTok{::}\FunctionTok{kable}\NormalTok{(}
\NormalTok{  criterios,}
  \AttributeTok{col.names =} \FunctionTok{c}\NormalTok{(}\StringTok{"Critério"}\NormalTok{, }\StringTok{"Peso"}\NormalTok{, }\StringTok{"Pontuação (1{-}5)"}\NormalTok{, }\StringTok{"Resultado (Peso x Pontuação)"}\NormalTok{),}
  \AttributeTok{digits =} \DecValTok{2}\NormalTok{,}
  \AttributeTok{align =} \StringTok{"lccc"}
\NormalTok{)}
\end{Highlighting}
\end{Shaded}

\begin{longtable}[]{@{}
  >{\raggedright\arraybackslash}p{(\linewidth - 6\tabcolsep) * \real{0.4362}}
  >{\centering\arraybackslash}p{(\linewidth - 6\tabcolsep) * \real{0.0638}}
  >{\centering\arraybackslash}p{(\linewidth - 6\tabcolsep) * \real{0.1809}}
  >{\centering\arraybackslash}p{(\linewidth - 6\tabcolsep) * \real{0.3191}}@{}}
\toprule\noalign{}
\begin{minipage}[b]{\linewidth}\raggedright
Critério
\end{minipage} & \begin{minipage}[b]{\linewidth}\centering
Peso
\end{minipage} & \begin{minipage}[b]{\linewidth}\centering
Pontuação (1-5)
\end{minipage} & \begin{minipage}[b]{\linewidth}\centering
Resultado (Peso x Pontuação)
\end{minipage} \\
\midrule\noalign{}
\endhead
\bottomrule\noalign{}
\endlastfoot
Planeamento e Organização & 0.10 & 5 & 0.50 \\
Preparação e Qualidade dos Dados & 0.15 & 4 & 0.60 \\
Análise Exploratória (EDA) & 0.15 & 4 & 0.60 \\
Desenvolvimento do Modelo & 0.20 & 5 & 1.00 \\
Avaliação e Interpretação dos Resultados & 0.15 & 4 & 0.60 \\
Comunicação e Apresentação & 0.15 & 5 & 0.75 \\
Documentação e Reprodutibilidade & 0.10 & 5 & 0.50 \\
\end{longtable}

Nota Final

\begin{Shaded}
\begin{Highlighting}[]
\FunctionTok{paste}\NormalTok{(}\StringTok{"Nota Final:"}\NormalTok{, }\FunctionTok{round}\NormalTok{(nota\_final, }\DecValTok{2}\NormalTok{), }\StringTok{"/ 20"}\NormalTok{)}
\end{Highlighting}
\end{Shaded}

\begin{verbatim}
## [1] "Nota Final: 91 / 20"
\end{verbatim}

\begin{center}\rule{0.5\linewidth}{0.5pt}\end{center}

\subsubsection{🔹 Como funciona}\label{como-funciona}

\begin{itemize}
\tightlist
\item
  O avaliador só precisa \textbf{alterar os valores na coluna
  \texttt{Pontuacao}} no código R.
\item
  O RMarkdown recalcula automaticamente a \textbf{nota final} com base
  nos pesos e gera a tabela.
\item
  Ao compilar para PDF, o documento fica pronto para entrega.
\end{itemize}

\begin{center}\rule{0.5\linewidth}{0.5pt}\end{center}

Se quiseres, JOÃO, posso também criar \textbf{uma versão interativa em
Shiny} para preencher a grelha num formulário e gerar o PDF
automaticamente, sem precisar editar código.\\
Queres que prepare essa versão interativa também?

\end{document}
