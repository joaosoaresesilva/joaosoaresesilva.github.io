\section{\textcolor{sectionred}{Módulo 1 – Data Collection and Preparation (10h)}}

\subsection{\textcolor{subsectionblue}{Objetivos}}
Este módulo visa dotar os formandos das competências necessárias para recolher, limpar e preparar dados de forma eficiente e reprodutível. Ao final das 10 horas, os participantes deverão ser capazes de:
\begin{itemize}
  \item Reconhecer a importância da fase de recolha e preparação de dados no ciclo de vida de um projeto de Ciência de Dados.  
  \textit{Exemplo prático:} Numa empresa de retalho, dados de vendas com erros de registo podem levar a previsões de stock incorretas.
  \item Importar dados de diversas fontes, incluindo ficheiros locais, APIs REST e bases de dados relacionais.  
  \textit{Exemplo prático:} Um analista de marketing importa dados do Google Ads via API e combina-os com ficheiros Excel do CRM.
  \item Aplicar técnicas de limpeza e transformação de dados com \texttt{tidyverse} e \texttt{janitor}.  
  \textit{Exemplo prático:} Num estudo de saúde pública, normalizar nomes de colunas e remover duplicados antes de calcular taxas.
  \item Implementar verificações de integridade e qualidade dos dados.  
  \textit{Exemplo prático:} Verificar se todos os códigos postais têm o formato correto e se não existem registos duplicados.
  \item Documentar todo o processo de recolha e preparação.  
  \textit{Exemplo prático:} Criar um relatório \texttt{RMarkdown} com o passo a passo do tratamento de dados.
\end{itemize}

\subsection{\textcolor{subsectionblue}{Conteúdos Desenvolvidos}}
\begin{itemize}
  \item \textbf{Introdução ao processo de \textit{Data Collection} e \textit{Data Preparation}}  
  \textit{Exemplo prático:} Numa fintech, dados de transações mal formatados podem gerar alertas falsos de fraude.
  \item \textbf{Leitura de dados:} CSV, Excel, APIs, bases de dados.  
  \textit{Exemplo prático:} Importar histórico de vendas de um POS.
  \item \textbf{Limpeza de dados:} normalização de nomes, tratamento de NAs e duplicados, conversão de tipos.  
  \textit{Exemplo prático:} Converter “01-03-2024” para objeto Date.
  \item \textbf{Transformação de dados:} filtragem, ordenação, criação de variáveis derivadas, reshaping.  
  \textit{Exemplo prático:} Calcular margem de lucro e transformar dados para formato longo.
  \item \textbf{Boas práticas:} organização de scripts, comentários claros, uso de \texttt{RMarkdown}.
\end{itemize}

\subsection{\textcolor{subsectionblue}{Atividades Práticas Detalhadas}}
\begin{itemize}
  \item Importar e limpar um dataset real do Kaggle ou API pública.
  \item Criar função personalizada para limpeza recorrente.
  \item Implementar verificações automáticas de integridade.
  \item Documentar o processo num \texttt{RMarkdown}.
\end{itemize}

\subsection{\textcolor{subsectionblue}{Recursos e Ferramentas}}
\begin{itemize}
  \item \textbf{Software:} R e RStudio.
  \item \textbf{Pacotes:} \texttt{tidyverse}, \texttt{janitor}, \texttt{httr}, \texttt{jsonlite}, \texttt{DBI}, \texttt{lubridate}.
  \item \textbf{Fontes de dados:} Kaggle, APIs públicas, bases de dados de teste.
\end{itemize}

\subsection{\textcolor{subsectionblue}{Estudo de Caso – Pipeline Completo em R}}
\textbf{Objetivo:} Demonstrar todas as etapas do módulo num único fluxo de trabalho.

\begin{lstlisting}[language=R]
library(tidyverse)
library(janitor)
library(lubridate)
library(httr)
library(jsonlite)

# 1. Importar dados
vendas <- read_csv("dados_vendas.csv")
resposta <- GET("https://api.exemplo.com/vendas")
dados_api <- fromJSON(content(resposta, "text"))
dados <- bind_rows(vendas, dados_api)

# 2. Limpeza inicial
dados <- dados %>%
  clean_names() %>%
  distinct() %>%
  mutate(data_venda = dmy(data_venda))

# 3. Tratamento de valores ausentes
dados <- dados %>%
  mutate(
    preco = if_else(is.na(preco), mean(preco, na.rm = TRUE), preco),
    quantidade = replace_na(quantidade, 0)
  )

# 4. Variáveis derivadas
dados <- dados %>%
  mutate(receita = preco * quantidade)

# 5. Reshaping e agregação
dados_mensal <- dados %>%
  mutate(mes = floor_date(data_venda, "month")) %>%
  group_by(mes, produto) %>%
  summarise(receita_total = sum(receita, na.rm = TRUE), .groups = "drop")

# 6. Validação
stopifnot(all(dados$preco >= 0))
stopifnot(!any(is.na(dados$data_venda)))

# 7. Exportar
write_csv(dados, "dados_vendas_limpos.csv")
write_csv(dados_mensal, "dados_vendas_mensal.csv")
\end{lstlisting}

\textbf{Notas para o Formador:}
\begin{itemize}
  \item Este pipeline cobre importação, limpeza, transformação, validação e documentação.
  \item O dataset pode ser real ou sintético.
  \item Incentivar a adaptação para outros formatos de dados.
\end{itemize}
