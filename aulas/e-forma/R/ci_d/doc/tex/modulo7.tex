\section{\textcolor{sectionred}{Módulo 7 – Tooling and Infrastructure (6h)}}

\subsection{\textcolor{subsectionblue}{Objetivos}}
Ao final deste módulo, os formandos deverão ser capazes de:
\begin{itemize}
  \item Configurar e manter ambientes de desenvolvimento eficientes para projetos de Ciência de Dados.  
  \item Automatizar tarefas e fluxos de trabalho para aumentar a produtividade.  
  \item Utilizar ferramentas de controlo de versões para gerir código e colaborar em equipa.  
  \item Integrar pipelines de dados e modelos em ambientes de produção.
\end{itemize}

\subsection{\textcolor{subsectionblue}{Conteúdos Desenvolvidos}}
\begin{itemize}
  \item \textbf{Organização de projetos:} estrutura de pastas e ficheiros; nomeação consistente de ficheiros e scripts.
  \item \textbf{Gestão de dependências:} uso do \texttt{renv} para isolar ambientes; ficheiros \texttt{DESCRIPTION} e \texttt{requirements.txt}.
  \item \textbf{Controlo de versões:} Git básico (commits, branches, merges); plataformas de colaboração (GitHub, GitLab).
  \item \textbf{Automação e pipelines:} scripts agendados (cron jobs, tasks); integração contínua (CI) e entrega contínua (CD).
\end{itemize}

\subsection{\textcolor{subsectionblue}{Atividades Práticas Detalhadas}}
\begin{itemize}
  \item Criar a estrutura de um projeto de Ciência de Dados com pastas e scripts organizados.
  \item Configurar um ambiente isolado com \texttt{renv} e instalar pacotes necessários.
  \item Criar um repositório Git e praticar operações básicas (commit, branch, merge).
  \item Configurar um script para ser executado automaticamente (agendamento).
\end{itemize}

\subsection{\textcolor{subsectionblue}{Recursos e Ferramentas}}
\begin{itemize}
  \item \textbf{Software:} R, RStudio, Git.
  \item \textbf{Pacotes:} \texttt{renv}, \texttt{targets}, \texttt{usethis}.
  \item \textbf{Plataformas:} GitHub, GitLab.
\end{itemize}

\subsection{\textcolor{subsectionblue}{Estudo de Caso – Estruturação e Automação de um Projeto}}
\textbf{Objetivo:} Demonstrar como criar um projeto organizado, gerir dependências e automatizar tarefas.

\begin{lstlisting}[language=R]
# 1. Criar estrutura de projeto
usethis::create_project("projeto_vendas")
dir.create("data_raw")
dir.create("data_processed")
dir.create("scripts")
dir.create("reports")

# 2. Iniciar controlo de versões
usethis::use_git()

# 3. Configurar ambiente isolado
install.packages("renv")
renv::init()

# 4. Script de importação e limpeza (scripts/01_importacao.R)
library(tidyverse)
dados <- read_csv("data_raw/vendas.csv") %>%
  janitor::clean_names() %>%
  filter(!is.na(valor_venda))
write_csv(dados, "data_processed/vendas_limpo.csv")

# 5. Automatizar execução diária (exemplo em sistema Unix com cron)
# Abrir crontab: crontab -e
# Adicionar linha para executar script às 2h da manhã:
# 0 2 * * * Rscript /caminho/projeto_vendas/scripts/01_importacao.R

# 6. Versionar alterações e sincronizar com GitHub
# git add .
# git commit -m "Estrutura inicial e script de importação"
# git push origin main
\end{lstlisting}

\textbf{Notas para o Formador:}
\begin{itemize}
  \item Este estudo de caso mostra como combinar organização, gestão de dependências, controlo de versões e automação.
  \item Incentivar os alunos a adaptar a estrutura e scripts às necessidades dos seus próprios projetos.
  \item Discutir a importância de manter um README atualizado com instruções de execução.
\end{itemize}
