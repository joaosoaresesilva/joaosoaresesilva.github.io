\section{\textcolor{sectionred}{Module 7 – Tooling and Infrastructure (6h)}}

\subsection{\textcolor{subsectionblue}{Objectives}}
By the end of this module, students should be able to:
\begin{itemize}
  \item Configure and maintain efficient development environments for Data Science projects.
  \item Automate tasks and workflows to increase productivity.
  \item Use version control tools to manage code and collaborate in a team.
  \item Integrate data and model pipelines into production environments.
\end{itemize}

\subsection{\textcolor{subsectionblue}{Developed Contents}}
\begin{itemize}
  \item \textbf{Project organization:} folder and file structure; consistent naming of files and scripts.
  \item \textbf{Dependency management:} use of \texttt{renv} to isolate environments; \texttt{DESCRIPTION} and \texttt{requirements.txt} files.
  \item \textbf{Version control:} Basic Git (commits, branches, merges); collaboration platforms (GitHub, GitLab).
  \item \textbf{Automation and pipelines:} scheduled scripts (cron jobs, tasks); continuous integration (CI) and continuous delivery (CD).
\end{itemize}

\subsection{\textcolor{subsectionblue}{Detailed Practical Activities}}
\begin{itemize}
  \item Create a Data Science project structure with organized folders and scripts.
  \item Configure an isolated environment with \texttt{renv} and install necessary packages.
  \item Create a Git repository and practice basic operations (commit, branch, merge).
  \item Configure a script to run automatically (scheduling).
\end{itemize}

\subsection{\textcolor{subsectionblue}{Resources and Tools}}
\begin{itemize}
  \item \textbf{Software:} R, RStudio, Git.
  \item \textbf{Packages:} \texttt{renv}, \texttt{targets}, \texttt{usethis}.
  \item \textbf{Platforms:} GitHub, GitLab.
\end{itemize}

\subsection{\textcolor{subsectionblue}{Case Study – Project Structuring and Automation}}
\textbf{Objective:} Demonstrate how to create an organized project, manage dependencies, and automate tasks.

\begin{lstlisting}[language=R]
# 1. Create project structure
usethis::create_project("sales_project")
dir.create("data_raw")
dir.create("data_processed")
dir.create("scripts")
dir.create("reports")

# 2. Initiate version control
usethis::use_git()

# 3. Configure isolated environment
install.packages("renv")
renv::init()

# 4. Import and cleaning script (scripts/01_importation.R)
library(tidyverse)
dados <- read_csv("data_raw/sales.csv") %>%
  janitor::clean_names() %>%
  filter(!is.na(sales_value))
write_csv(dados, "data_processed/clean_sales.csv")

# 5. Automate daily execution (example on Unix system with cron)
# Open crontab: crontab -e
# Add line to run script at 2 a.m.:
# 0 2 * * * Rscript /path/to/sales_project/scripts/01_importation.R

# 6. Version changes and sync with GitHub
# git add .
# git commit -m "Initial structure and import script"
# git push origin main
\end{lstlisting}

\textbf{Notes for the Instructor:}
\begin{itemize}
  \item This case study shows how to combine organization, dependency management, version control, and automation.
  \item Encourage students to adapt the structure and scripts to their own project needs.
  \item Discuss the importance of keeping a README updated with execution instructions.
\end{itemize}