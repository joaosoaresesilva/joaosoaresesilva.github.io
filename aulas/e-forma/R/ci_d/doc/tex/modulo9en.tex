\section{\textcolor{sectionred}{Module 9 – Evaluation (4h)}}

\subsection{\textcolor{subsectionblue}{Objectives}}
By the end of this module, students should be able to:
\begin{itemize}
  \item Define evaluation criteria and instruments suitable for the course objectives.
  \item Apply different types of evaluation (diagnostic, formative, and summative) throughout the course.
  \item Provide constructive feedback aimed at continuous improvement.
  \item Use digital tools to collect, record, and analyze evaluation results.
  \item Document the evaluation process to ensure transparency and traceability.
\end{itemize}

\subsection{\textcolor{subsectionblue}{Developed Contents}}
\begin{itemize}
  \item \textbf{Types of evaluation:} diagnostic, formative, and summative.
  \item \textbf{Performance criteria and indicators:} clarity, objectivity, and alignment with course objectives.
  \item \textbf{Evaluation instruments:} evaluation grids, rubrics, tests, presentations; digital tools (Google Forms, Microsoft Forms, Moodle).
  \item \textbf{Constructive feedback:} "strengths – areas for improvement – suggestions" structure; assertive communication techniques.
  \item \textbf{Recording and documentation:} secure storage of results; evaluation reports and feedback meeting minutes.
\end{itemize}

\subsection{\textcolor{subsectionblue}{Detailed Practical Activities}}
\begin{itemize}
  \item Create an evaluation grid for a final project, with defined criteria and weights.
  \item Apply the grid to a sample work and calculate the final grade.
  \item Prepare a feedback report based on the results.
  \item Simulate the use of a digital tool to collect and analyze results.
\end{itemize}

\subsection{\textcolor{subsectionblue}{Resources and Tools}}
\begin{itemize}
  \textbf{Software:} R and RStudio, Google Forms, Microsoft Forms, Moodle.
  \textbf{R Packages:} \texttt{dplyr}, \texttt{readr}, \texttt{knitr}.
  \textbf{Documents:} templates for grids and rubrics.
\end{itemize}

\subsection{\textcolor{subsectionblue}{Case Study – Evaluation Grid in R}}
\textbf{Objective:} Demonstrate how to create and apply an evaluation grid with automatic final grade calculation.

\begin{lstlisting}[language=R]
library(dplyr)

# 1. Define criteria and weights
criterios <- data.frame(
  criterio = c("Technical Quality", "Creativity", "Communication", "Documentation"),
  peso = c(0.4, 0.2, 0.2, 0.2),
  pontuacao = c(4, 5, 4, 3) # example evaluation (1 to 5)
)

# 2. Calculate final grade (0-20 scale)
criterios <- criterios %>%
  mutate(resultado = peso * pontuacao)

nota_final <- sum(criterios$resultado) / sum(criterios$peso) * 4 # 5 points = 20 values

# 3. Show results
criterios
nota_final

# 4. Export simple report
library(knitr)
kable(criterios, col.names = c("Criterion", "Weight", "Score", "Result"))
cat("Final Grade:", round(nota_final, 2), "/ 20")
\end{lstlisting}

\textbf{Notes for the Instructor:}
\begin{itemize}
  \item This example shows how to use R to automate grade calculation based on a grid.
  \item Encourage students to adapt criteria and weights to the type of project.
  \item Discuss the importance of keeping clear and accessible records for auditing.
\end{itemize}