\section{\textcolor{sectionred}{Módulo 5 – Research and Learning (6h)}}

\subsection{\textcolor{subsectionblue}{Objetivos}}
Ao final deste módulo, os formandos deverão ser capazes de:
\begin{itemize}
  \item Desenvolver competências de pesquisa contínua para se manterem atualizados nas áreas de Ciência de Dados, Machine Learning e Inteligência Artificial.  
  \item Avaliar criticamente novas bibliotecas, técnicas e metodologias antes de as adotar.  
  \item Integrar novas ferramentas e abordagens nos fluxos de trabalho existentes.  
  \item Partilhar conhecimento com a comunidade e com a equipa de trabalho.
\end{itemize}

\subsection{\textcolor{subsectionblue}{Conteúdos Desenvolvidos}}
\begin{itemize}
  \item \textbf{Fontes de informação e atualização:} repositórios de pacotes (CRAN, Bioconductor), blogs e comunidades (\texttt{R-bloggers}, Stack Overflow, Posit Community), publicações científicas (arXiv, IEEE, Nature Machine Intelligence).
  \item \textbf{Avaliação de novas ferramentas:} critérios de avaliação (desempenho, documentação, comunidade, manutenção), testes de integração em projetos piloto.
  \item \textbf{Integração no fluxo de trabalho:} adaptação de código e pipelines, automação de tarefas repetitivas.
  \item \textbf{Partilha de conhecimento:} documentação interna e externa, contribuição para projetos open-source.
\end{itemize}

\subsection{\textcolor{subsectionblue}{Atividades Práticas Detalhadas}}
\begin{itemize}
  \item Pesquisar e apresentar um pacote R recente, explicando as suas funcionalidades e casos de uso.
  \item Implementar um exemplo prático com uma técnica ou pacote recém-descoberto.
  \item Criar um documento comparativo entre duas abordagens para o mesmo problema.
  \item Publicar um pequeno tutorial ou exemplo de código num repositório GitHub.
\end{itemize}

\subsection{\textcolor{subsectionblue}{Recursos e Ferramentas}}
\begin{itemize}
  \item \textbf{Software:} R e RStudio.
  \item \textbf{Pacotes:} variáveis conforme a pesquisa (ex.: \texttt{tidyverse}, \texttt{data.table}, \texttt{lightgbm}).
  \item \textbf{Fontes de dados:} CRAN, GitHub, arXiv, APIs públicas.
\end{itemize}

\subsection{\textcolor{subsectionblue}{Estudo de Caso – Pesquisa e Integração de um Novo Pacote}}
\textbf{Objetivo:} Demonstrar como identificar, testar e integrar um novo pacote R num fluxo de trabalho.

\begin{lstlisting}[language=R]
# 1. Pesquisa de pacote no CRAN
# Exemplo: pacote 'data.table' para manipulação eficiente de dados

# 2. Instalação e carregamento
install.packages("data.table")
library(data.table)

# 3. Comparação de desempenho com dplyr
library(tidyverse)
dados <- as.data.frame(matrix(runif(1e6), ncol = 10))

# Usando dplyr
system.time({
  df_dplyr <- as_tibble(dados) %>%
    summarise(across(everything(), mean))
})

# Usando data.table
system.time({
  dt <- as.data.table(dados)
  dt_means <- dt[, lapply(.SD, mean)]
})

# 4. Integração no fluxo de trabalho
# Substituir operações lentas por equivalentes em data.table
# Documentar alterações e ganhos de desempenho

# 5. Partilha de resultados
# Criar um README com instruções e benchmarks
\end{lstlisting}

\textbf{Notas para o Formador:}
\begin{itemize}
  \item Incentivar os alunos a escolher pacotes relevantes para os seus interesses ou área de trabalho.
  \item Discutir critérios objetivos para adoção de novas ferramentas.
  \item Mostrar como documentar e comunicar resultados de testes à equipa.
\end{itemize}
