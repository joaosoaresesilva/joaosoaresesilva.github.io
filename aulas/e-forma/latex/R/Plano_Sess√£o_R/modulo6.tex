\section{\textcolor{sectionred}{Módulo 6 – Ad-hoc Analysis (8h)}}

\subsection{\textcolor{subsectionblue}{Objetivos}}
Ao final deste módulo, os formandos deverão ser capazes de:
\begin{itemize}
  \item Compreender o papel das análises \textit{ad-hoc} na resposta rápida a questões específicas de negócio ou investigação.  
  \item Formular hipóteses e estruturar análises direcionadas para responder a perguntas concretas.  
  \item Selecionar e aplicar métodos estatísticos e de visualização adequados ao problema.  
  \item Comunicar resultados de forma clara e objetiva, focando nas conclusões mais relevantes.
\end{itemize}

\subsection{\textcolor{subsectionblue}{Conteúdos Desenvolvidos}}
\begin{itemize}
  \item \textbf{Definição e contexto de análises \textit{ad-hoc}} — diferença entre análises planeadas e \textit{ad-hoc}; quando e porquê utilizá-las.
  \item \textbf{Formulação de hipóteses e perguntas de negócio} — transformar questões vagas em hipóteses testáveis.
  \item \textbf{Seleção de métodos e ferramentas} — escolha de técnicas estatísticas, filtros e visualizações adequadas.
  \item \textbf{Execução rápida e validação de resultados} — garantir que a rapidez não compromete a qualidade.
  \item \textbf{Comunicação e entrega} — formatos de entrega: relatório breve, e-mail, apresentação curta.
\end{itemize}

\subsection{\textcolor{subsectionblue}{Atividades Práticas Detalhadas}}
\begin{itemize}
  \item Receber uma questão de negócio simulada e desenvolver uma análise \textit{ad-hoc} para respondê-la.
  \item Criar visualizações simples e eficazes para comunicar resultados.
  \item Aplicar um teste estatístico adequado ao problema apresentado.
  \item Documentar a análise num formato conciso e claro.
\end{itemize}

\subsection{\textcolor{subsectionblue}{Recursos e Ferramentas}}
\begin{itemize}
  \item \textbf{Software:} R e RStudio.
  \item \textbf{Pacotes:} \texttt{tidyverse}, \texttt{lubridate}, \texttt{ggplot2}, \texttt{broom}.
  \item \textbf{Fontes de dados:} datasets internos ou públicos simulando cenários reais.
\end{itemize}

\subsection{\textcolor{subsectionblue}{Estudo de Caso – Análise \textit{Ad-hoc} em R}}
\textbf{Objetivo:} Demonstrar como responder rapidamente a uma questão específica com dados disponíveis.

\begin{lstlisting}[language=R]
library(tidyverse)
library(lubridate)
library(broom)

# Cenário: O gestor quer saber se a média de vendas do último mês
# foi significativamente diferente do mês anterior.

# 1. Importar dados
dados <- read_csv("vendas.csv") %>%
  mutate(data = dmy(data))

# 2. Filtrar últimos dois meses
ultimo_mes <- max(month(dados$data))
dados_filtrados <- dados %>%
  filter(month(data) %in% c(ultimo_mes, ultimo_mes - 1))

# 3. Estatísticas descritivas
dados_filtrados %>%
  group_by(mes = month(data)) %>%
  summarise(media_vendas = mean(vendas, na.rm = TRUE),
            sd_vendas = sd(vendas, na.rm = TRUE))

# 4. Teste estatístico (t-test)
resultado_t <- t.test(vendas ~ month(data), data = dados_filtrados)
tidy(resultado_t)

# 5. Visualização rápida
ggplot(dados_filtrados, aes(x = factor(month(data)), y = vendas)) +
  geom_boxplot(fill = "skyblue") +
  labs(x = "Mês", y = "Vendas", title = "Comparação de Vendas entre Meses")

# 6. Conclusão
# Se p-valor < 0.05, há diferença estatisticamente significativa.
\end{lstlisting}

\textbf{Notas para o Formador:}
\begin{itemize}
  \item Este exemplo mostra como estruturar rapidamente uma análise desde a importação de dados até à conclusão.
  \item Incentivar os alunos a adaptar o código a outros tipos de questões \textit{ad-hoc}.
  \item Discutir a importância de comunicar apenas o essencial em análises urgentes.
\end{itemize}
