\section{\textcolor{sectionred}{Módulo 8 – Deployment e Manutenção de Modelos (6h)}}

\subsection{\textcolor{subsectionblue}{Objetivos}}
Ao final deste módulo, os formandos deverão ser capazes de:
\begin{itemize}
  \item Compreender o processo de disponibilização de modelos de Machine Learning em ambientes de produção.  
  \item Implementar APIs para servir previsões de modelos em tempo real ou em lote.  
  \item Monitorizar o desempenho de modelos em produção e detetar \textit{data drift} ou degradação de performance.  
  \item Realizar manutenção e re-treino periódico de modelos para garantir relevância e precisão.  
  \item Documentar e versionar modelos para rastreabilidade e auditoria.
\end{itemize}

\subsection{\textcolor{subsectionblue}{Conteúdos Desenvolvidos}}
\begin{itemize}
  \item \textbf{Conceitos de deployment:} diferença entre ambientes de desenvolvimento, teste e produção; modos de disponibilização (batch, tempo real, \textit{edge deployment}).
  \item \textbf{Servir modelos via API:} uso do pacote \texttt{plumber} para criar APIs REST em R; segurança e autenticação de APIs.
  \item \textbf{Monitorização de modelos:} métricas de desempenho em produção; deteção de \textit{data drift} e \textit{concept drift}.
  \item \textbf{Manutenção e re-treino:} estratégias de re-treino (agendado, baseado em eventos); automação de pipelines de re-treino.
  \item \textbf{Versionamento e documentação de modelos:} armazenamento de modelos com \texttt{pins} ou \texttt{modeltime}; registo de alterações e auditoria.
\end{itemize}

\subsection{\textcolor{subsectionblue}{Atividades Práticas Detalhadas}}
\begin{itemize}
  \item Criar uma API com \texttt{plumber} para servir previsões de um modelo treinado.
  \item Implementar um script de monitorização que regista métricas de desempenho diariamente.
  \item Simular \textit{data drift} e executar re-treino automático.
  \item Documentar o processo de deployment e manutenção num relatório técnico.
\end{itemize}

\subsection{\textcolor{subsectionblue}{Recursos e Ferramentas}}
\begin{itemize}
  \item \textbf{Software:} R e RStudio.
  \item \textbf{Pacotes:} \texttt{plumber}, \texttt{pins}, \texttt{modeltime}, \texttt{tidyverse}.
  \item \textbf{Plataformas:} Servidores locais, \texttt{shinyapps.io}, Docker.
\end{itemize}

\subsection{\textcolor{subsectionblue}{Estudo de Caso – API de Previsões com \texttt{plumber}}}
\textbf{Objetivo:} Demonstrar como disponibilizar um modelo treinado via API REST.

\begin{lstlisting}[language=R]
# 1. Treinar e guardar modelo
library(tidyverse)
library(caret)
dados <- read_csv("dados_clientes.csv")

modelo <- train(churn ~ ., data = dados, method = "glm", family = "binomial")
saveRDS(modelo, "modelo_churn.rds")

# 2. Criar API com plumber (ficheiro: api.R)
library(plumber)

#* @apiTitle API de Previsão de Churn
#* @param idade:int Idade do cliente
#* @param rendimento:double Rendimento anual
#* @post /prever
function(idade, rendimento){
  modelo <- readRDS("modelo_churn.rds")
  novo_dado <- data.frame(idade = as.integer(idade),
                          rendimento = as.numeric(rendimento))
  prob <- predict(modelo, novo_dado, type = "prob")[,2]
  list(probabilidade_churn = prob)
}

# 3. Executar API
# plumber::pr("api.R") %>% pr_run(port = 8000)

# 4. Testar API
# POST para http://localhost:8000/prever com JSON:
# {"idade": 35, "rendimento": 45000}
\end{lstlisting}

\textbf{Notas para o Formador:}
\begin{itemize}
  \item Este exemplo mostra um fluxo simples de deployment com \texttt{plumber}.
  \item Incentivar os alunos a adicionar autenticação e logging à API.
  \item Discutir como integrar esta API num sistema maior (ex.: dashboard, aplicação web).
\end{itemize}
