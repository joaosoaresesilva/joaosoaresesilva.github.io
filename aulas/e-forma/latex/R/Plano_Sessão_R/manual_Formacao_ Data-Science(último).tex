\documentclass[12pt,a4paper]{article}
\usepackage[utf8]{inputenc}
\usepackage[T1]{fontenc}
\usepackage[table]{xcolor}
\usepackage{geometry}
\usepackage{titlesec}
\usepackage{hyperref}
\usepackage{listings}
\usepackage{xcolor}
\usepackage[portuguese]{babel}

\geometry{margin=2.5cm}

% Definir cores para títulos
\definecolor{sectionred}{RGB}{200,0,0}
\definecolor{subsectionblue}{RGB}{0,0,180}

\titleformat{\section}
  {\color{sectionred}\normalfont\Large\bfseries}
  {\thesection}{1em}{}
\titleformat{\subsection}
  {\color{subsectionblue}\normalfont\large\bfseries}
  {\thesubsection}{1em}{}

% Configuração para mostrar código R
\lstdefinelanguage{R}{
  keywords={library,read_csv,ggplot,geom_col,labs,aes,geom_bar,stat,identity},
  sensitive=true,
  comment=[l]{\#},
  morestring=[b]",
}
\lstset{
  language=R,
  basicstyle=\ttfamily\small,
  keywordstyle=\color{blue},
  commentstyle=\color{gray},
  stringstyle=\color{red},
  showstringspaces=false,
  breaklines=true
}

\title{\textbf{Manual de Formação em Ciência de Dados com R}}
\author{Formador: JOÃO SOARES E SILVA}
\date{\today}

\begin{document}

\maketitle
\tableofcontents
\newpage

% =========================
% Módulo 1
% =========================
\section{\textcolor{sectionred}{Módulo 1 – Data Collection and Preparation (10h)}}

\subsection{\textcolor{subsectionblue}{Objetivos}}
Este módulo visa dotar os formandos das competências necessárias para recolher, limpar e preparar dados de forma eficiente e reprodutível. Ao final das 10 horas, os participantes deverão ser capazes de:
\begin{itemize}
  \item Reconhecer a importância da fase de recolha e preparação de dados no ciclo de vida de um projeto de Ciência de Dados.  
  \textit{Exemplo prático:} Numa empresa de retalho, dados de vendas com erros de registo podem levar a previsões de stock incorretas.
  \item Importar dados de diversas fontes, incluindo ficheiros locais, APIs REST e bases de dados relacionais.  
  \textit{Exemplo prático:} Um analista de marketing importa dados do Google Ads via API e combina-os com ficheiros Excel do CRM.
  \item Aplicar técnicas de limpeza e transformação de dados com \texttt{tidyverse} e \texttt{janitor}.  
  \textit{Exemplo prático:} Num estudo de saúde pública, normalizar nomes de colunas e remover duplicados antes de calcular taxas.
  \item Implementar verificações de integridade e qualidade dos dados.  
  \textit{Exemplo prático:} Verificar se todos os códigos postais têm o formato correto e se não existem registos duplicados.
  \item Documentar todo o processo de recolha e preparação.  
  \textit{Exemplo prático:} Criar um relatório \texttt{RMarkdown} com o passo a passo do tratamento de dados.
\end{itemize}

\subsection{\textcolor{subsectionblue}{Conteúdos Desenvolvidos}}
\begin{itemize}
  \item \textbf{Introdução ao processo de \textit{Data Collection} e \textit{Data Preparation}}  
  \textit{Exemplo prático:} Numa fintech, dados de transações mal formatados podem gerar alertas falsos de fraude.
  \item \textbf{Leitura de dados:} CSV, Excel, APIs, bases de dados.  
  \textit{Exemplo prático:} Importar histórico de vendas de um POS.
  \item \textbf{Limpeza de dados:} normalização de nomes, tratamento de NAs e duplicados, conversão de tipos.  
  \textit{Exemplo prático:} Converter “01-03-2024” para objeto Date.
  \item \textbf{Transformação de dados:} filtragem, ordenação, criação de variáveis derivadas, reshaping.  
  \textit{Exemplo prático:} Calcular margem de lucro e transformar dados para formato longo.
  \item \textbf{Boas práticas:} organização de scripts, comentários claros, uso de \texttt{RMarkdown}.
\end{itemize}

\subsection{\textcolor{subsectionblue}{Atividades Práticas Detalhadas}}
\begin{itemize}
  \item Importar e limpar um dataset real do Kaggle ou API pública.
  \item Criar função personalizada para limpeza recorrente.
  \item Implementar verificações automáticas de integridade.
  \item Documentar o processo num \texttt{RMarkdown}.
\end{itemize}

\subsection{\textcolor{subsectionblue}{Recursos e Ferramentas}}
\begin{itemize}
  \item \textbf{Software:} R e RStudio.
  \item \textbf{Pacotes:} \texttt{tidyverse}, \texttt{janitor}, \texttt{httr}, \texttt{jsonlite}, \texttt{DBI}, \texttt{lubridate}.
  \item \textbf{Fontes de dados:} Kaggle, APIs públicas, bases de dados de teste.
\end{itemize}

\subsection{\textcolor{subsectionblue}{Estudo de Caso – Pipeline Completo em R}}
\textbf{Objetivo:} Demonstrar todas as etapas do módulo num único fluxo de trabalho.

\begin{lstlisting}[language=R]
library(tidyverse)
library(janitor)
library(lubridate)
library(httr)
library(jsonlite)

# 1. Importar dados
vendas <- read_csv("dados_vendas.csv")
resposta <- GET("https://api.exemplo.com/vendas")
dados_api <- fromJSON(content(resposta, "text"))
dados <- bind_rows(vendas, dados_api)

# 2. Limpeza inicial
dados <- dados %>%
  clean_names() %>%
  distinct() %>%
  mutate(data_venda = dmy(data_venda))

# 3. Tratamento de valores ausentes
dados <- dados %>%
  mutate(
    preco = if_else(is.na(preco), mean(preco, na.rm = TRUE), preco),
    quantidade = replace_na(quantidade, 0)
  )

# 4. Variáveis derivadas
dados <- dados %>%
  mutate(receita = preco * quantidade)

# 5. Reshaping e agregação
dados_mensal <- dados %>%
  mutate(mes = floor_date(data_venda, "month")) %>%
  group_by(mes, produto) %>%
  summarise(receita_total = sum(receita, na.rm = TRUE), .groups = "drop")

# 6. Validação
stopifnot(all(dados$preco >= 0))
stopifnot(!any(is.na(dados$data_venda)))

# 7. Exportar
write_csv(dados, "dados_vendas_limpos.csv")
write_csv(dados_mensal, "dados_vendas_mensal.csv")
\end{lstlisting}

\textbf{Notas para o Formador:}
\begin{itemize}
  \item Este pipeline cobre importação, limpeza, transformação, validação e documentação.
  \item O dataset pode ser real ou sintético.
  \item Incentivar a adaptação para outros formatos de dados.
\end{itemize}


% =========================
% Módulo 2
% =========================
\section{\textcolor{sectionred}{Módulo 2 – Exploratory Data Analysis (EDA) (10h)}}

\subsection{\textcolor{subsectionblue}{Objetivos}}
Ao final deste módulo, os formandos deverão ser capazes de:
\begin{itemize}
  \item Compreender o papel da Análise Exploratória de Dados (EDA) no ciclo de vida de um projeto de Ciência de Dados.  
  \textit{Exemplo prático:} Antes de criar um modelo de previsão de vendas, uma empresa analisa a distribuição das vendas por região para identificar padrões sazonais.
  \item Explorar dados utilizando estatísticas descritivas e visualizações gráficas para obter uma compreensão inicial do conjunto de dados.  
  \textit{Exemplo prático:} Um hospital calcula médias e desvios padrão de tempos de espera para identificar gargalos no atendimento.
  \item Identificar padrões, tendências, outliers e relações entre variáveis.  
  \textit{Exemplo prático:} Uma seguradora deteta clientes com valores de indemnização anormalmente altos, investigando possíveis fraudes.
  \item Formular hipóteses iniciais com base nas observações obtidas.  
  \textit{Exemplo prático:} Após observar que clientes mais jovens compram mais online, uma loja formula a hipótese de que campanhas digitais terão maior impacto nesse grupo.
  \item Comunicar de forma clara os principais achados da análise exploratória.  
  \textit{Exemplo prático:} Um analista apresenta à direção um dashboard com gráficos interativos mostrando tendências de vendas por produto e região.
\end{itemize}

\subsection{\textcolor{subsectionblue}{Conteúdos Desenvolvidos}}
\begin{itemize}
  \item \textbf{Introdução à EDA e sua importância}  
  \textit{Exemplo prático:} Numa startup de mobilidade, a EDA revelou que a maioria das viagens curtas ocorre em horários de pico.
  \item \textbf{Estatísticas descritivas:} média, mediana, moda, variância, desvio padrão, amplitude, distribuições de frequência e percentis.  
  \textit{Exemplo prático:} Calcular a mediana de salários para evitar distorções causadas por valores extremos.
  \item \textbf{Visualização de dados:} histogramas, boxplots, scatterplots, gráficos de barras com \texttt{ggplot2}; visualizações interativas com \texttt{plotly}; mapas de calor e matrizes de correlação com \texttt{corrplot}.  
  \textit{Exemplo prático:} Criar um histograma da idade dos clientes para segmentar campanhas.
  \item \textbf{Identificação de outliers:} métodos gráficos e estatísticos (IQR, Z-score).  
  \textit{Exemplo prático:} Detetar transações bancárias suspeitas com valores muito acima da média.
  \item \textbf{Análise de correlação:} Pearson, Spearman, Kendall e interpretação de coeficientes.  
  \textit{Exemplo prático:} Avaliar a relação entre temperatura e consumo de gelados.
  \item \textbf{Boas práticas na apresentação de resultados:} gráficos claros, evitar sobrecarga de informação, contextualizar achados.  
  \textit{Exemplo prático:} Apresentar apenas as 5 variáveis mais relevantes num relatório executivo.
\end{itemize}

\subsection{\textcolor{subsectionblue}{Atividades Práticas Detalhadas}}
\begin{itemize}
  \item Calcular estatísticas descritivas para um dataset real.
  \item Criar visualizações com \texttt{ggplot2} para explorar distribuições e relações.
  \item Utilizar \texttt{plotly} para criar gráficos interativos.
  \item Gerar e interpretar uma matriz de correlação com \texttt{corrplot}.
  \item Identificar e documentar padrões, outliers e correlações relevantes.
\end{itemize}

\subsection{\textcolor{subsectionblue}{Recursos e Ferramentas}}
\begin{itemize}
  \item \textbf{Software:} R e RStudio.
  \item \textbf{Pacotes:} \texttt{ggplot2}, \texttt{plotly}, \texttt{corrplot}, \texttt{dplyr}.
  \item \textbf{Datasets de apoio:} vendas online, dados de saúde pública, dados meteorológicos.
\end{itemize}

\subsection{\textcolor{subsectionblue}{Estudo de Caso – EDA Completa em R}}
\textbf{Objetivo:} Demonstrar todas as etapas da EDA num único fluxo de trabalho.

\begin{lstlisting}[language=R]
library(tidyverse)
library(plotly)
library(corrplot)

# 1. Importar dataset
dados <- read_csv("vendas.csv")

# 2. Estatísticas descritivas
summary(dados)
dados %>% summarise(
  media_preco = mean(preco, na.rm = TRUE),
  mediana_preco = median(preco, na.rm = TRUE),
  desvio_padrao_preco = sd(preco, na.rm = TRUE)
)

# 3. Visualizações básicas
ggplot(dados, aes(x = preco)) + geom_histogram(binwidth = 5)
ggplot(dados, aes(x = categoria, y = preco)) + geom_boxplot()

# 4. Visualização interativa
grafico_interativo <- ggplot(dados, aes(x = preco, y = quantidade, color = categoria)) +
  geom_point()
ggplotly(grafico_interativo)

# 5. Matriz de correlação
matriz <- cor(dados %>% select_if(is.numeric), use = "complete.obs")
corrplot(matriz, method = "color", type = "upper", tl.col = "black")

# 6. Identificação de outliers (IQR)
Q1 <- quantile(dados$preco, 0.25, na.rm = TRUE)
Q3 <- quantile(dados$preco, 0.75, na.rm = TRUE)
IQR <- Q3 - Q1
outliers <- dados %>% filter(preco < (Q1 - 1.5*IQR) | preco > (Q3 + 1.5*IQR))
\end{lstlisting}

\textbf{Notas para o Formador:}
\begin{itemize}
  \item O dataset \texttt{vendas.csv} pode ser real ou sintético, mas deve conter variáveis numéricas e categóricas.
  \item Incentivar os alunos a experimentar diferentes tipos de gráficos e parâmetros.
  \item Discutir como as descobertas da EDA podem influenciar a modelagem subsequente.
\end{itemize}


% =========================
% Módulo 3
% =========================
\section{\textcolor{sectionred}{Módulo 3 – Model Development Support (12h)}}

\subsection{\textcolor{subsectionblue}{Objetivos}}
Ao final deste módulo, os formandos deverão ser capazes de:
\begin{itemize}
  \item Compreender o papel do desenvolvimento de modelos no ciclo de vida de um projeto de Ciência de Dados.  
  \textit{Exemplo prático:} Uma empresa de logística desenvolve um modelo para prever atrasos nas entregas com base em dados históricos e meteorológicos.
  \item Apoiar na criação de modelos preditivos supervisionados e não supervisionados.  
  \item Realizar engenharia de variáveis (\textit{feature engineering}) para melhorar a performance dos modelos.  
  \item Selecionar e aplicar algoritmos adequados ao tipo de problema.  
  \item Avaliar modelos utilizando métricas apropriadas e técnicas de validação.  
  \item Documentar e comunicar o processo de modelagem e os resultados obtidos.
\end{itemize}

\subsection{\textcolor{subsectionblue}{Conteúdos Desenvolvidos}}
\begin{itemize}
  \item \textbf{Introdução ao fluxo de trabalho de modelagem preditiva} — preparação de dados, treino, validação, teste e implementação.
  \item \textbf{Preparação dos dados:} divisão treino/validação/teste; normalização e padronização; codificação de variáveis categóricas.
  \item \textbf{Engenharia de variáveis:} criação de novas variáveis; seleção de variáveis relevantes; redução de dimensionalidade (PCA).
  \item \textbf{Treino de modelos:} uso de \texttt{caret}, \texttt{tidymodels} e \texttt{recipes}; algoritmos comuns (regressão linear/logística, árvores, random forest, k-NN).
  \item \textbf{Avaliação de modelos:} métricas para regressão (RMSE, MAE, $R^2$) e classificação (Acurácia, Precisão, Recall, F1-score, AUC); validação cruzada.
  \item \textbf{Comparação e seleção de modelos}.
\end{itemize}

\subsection{\textcolor{subsectionblue}{Atividades Práticas Detalhadas}}
\begin{itemize}
  \item Criar um modelo de regressão para prever preços de imóveis utilizando \texttt{caret}.
  \item Desenvolver um modelo de classificação para prever churn de clientes com \texttt{tidymodels}.
  \item Implementar um \texttt{recipe} para normalizar dados, criar variáveis derivadas e codificar variáveis categóricas.
  \item Comparar o desempenho de pelo menos dois algoritmos diferentes para o mesmo problema.
  \item Documentar o processo de modelagem e apresentar os resultados com métricas e gráficos.
\end{itemize}

\subsection{\textcolor{subsectionblue}{Recursos e Ferramentas}}
\begin{itemize}
  \item \textbf{Software:} R e RStudio.
  \item \textbf{Pacotes:} \texttt{caret}, \texttt{tidymodels}, \texttt{recipes}, \texttt{ggplot2}, \texttt{dplyr}.
  \item \textbf{Datasets:} preços de imóveis, churn de clientes, datasets públicos do UCI Machine Learning Repository.
\end{itemize}

\subsection{\textcolor{subsectionblue}{Estudo de Caso – Desenvolvimento de Modelo em R}}
\textbf{Objetivo:} Demonstrar todas as etapas do desenvolvimento de um modelo preditivo.

\begin{lstlisting}[language=R]
library(tidyverse)
library(caret)
library(tidymodels)

# 1. Importar dataset
dados <- read_csv("precos_imoveis.csv")

# 2. Divisão treino/teste
set.seed(123)
divisao <- initial_split(dados, prop = 0.7)
treino <- training(divisao)
teste <- testing(divisao)

# 3. Recipe de pré-processamento
receita <- recipe(preco ~ ., data = treino) %>%
  step_normalize(all_numeric_predictors()) %>%
  step_dummy(all_nominal_predictors())

# 4. Modelo regressão linear
modelo_rl <- linear_reg() %>% set_engine("lm")

# 5. Workflow
workflow_rl <- workflow() %>%
  add_recipe(receita) %>%
  add_model(modelo_rl)

# 6. Treinar
modelo_treinado <- fit(workflow_rl, data = treino)

# 7. Avaliar
predicoes <- predict(modelo_treinado, teste) %>%
  bind_cols(teste)
metrics(predicoes, truth = preco, estimate = .pred)

# 8. Comparar com Random Forest
modelo_rf <- rand_forest() %>%
  set_engine("ranger") %>%
  set_mode("regression")
workflow_rf <- workflow() %>%
  add_recipe(receita) %>%
  add_model(modelo_rf)
modelo_rf_treinado <- fit(workflow_rf, data = treino)
predicoes_rf <- predict(modelo_rf_treinado, teste) %>%
  bind_cols(teste)
metrics(predicoes_rf, truth = preco, estimate = .pred)
\end{lstlisting}

\textbf{Notas para o Formador:}
\begin{itemize}
  \item Este estudo de caso cobre preparação de dados, treino, avaliação e comparação de algoritmos.
  \item O dataset pode ser obtido de fontes abertas ou gerado sinteticamente.
  \item Incentivar a experimentação com diferentes algoritmos e hiperparâmetros.
  \item Discutir a escolha de métricas adequadas ao problema.
\end{itemize}


% =========================
% Módulo 4
% =========================
\section{\textcolor{sectionred}{Módulo 4 – Reporting and Documentation (8h)}}

\subsection{\textcolor{subsectionblue}{Objetivos}}
Ao final deste módulo, os formandos deverão ser capazes de:
\begin{itemize}
  \item Compreender a importância da comunicação clara e estruturada dos resultados em projetos de Ciência de Dados.  
  \item Criar relatórios técnicos e executivos que transmitam de forma eficaz as descobertas e conclusões.  
  \item Desenvolver dashboards interativos para visualização e monitorização de métricas e indicadores.  
  \item Adaptar a comunicação para diferentes públicos-alvo (técnico e não técnico).  
  \item Documentar o código, processos e decisões para garantir reprodutibilidade e manutenção futura.
\end{itemize}

\subsection{\textcolor{subsectionblue}{Conteúdos Desenvolvidos}}
\begin{itemize}
  \item \textbf{Boas práticas de comunicação de resultados em Data Science} — estruturar mensagens de forma clara, objetiva e visualmente apelativa.
  \item \textbf{Relatórios com \texttt{RMarkdown}:} estrutura de um documento técnico; inclusão de código, tabelas e gráficos; exportação para HTML, PDF e Word.
  \item \textbf{Dashboards interativos:} criação com \texttt{flexdashboard}; aplicações web com \texttt{shiny}; integração de visualizações dinâmicas (\texttt{plotly}, \texttt{leaflet}).
  \item \textbf{Storytelling com dados:} estrutura narrativa para apresentação de insights; uso de visualizações para reforçar mensagens-chave.
  \item \textbf{Documentação de código e processos:} comentários claros e consistentes; ficheiros \texttt{README} e guias de utilização; versionamento com Git/GitHub.
\end{itemize}

\subsection{\textcolor{subsectionblue}{Atividades Práticas Detalhadas}}
\begin{itemize}
  \item Criar um relatório técnico em \texttt{RMarkdown} com análise exploratória e resultados de um modelo preditivo.
  \item Desenvolver um dashboard interativo com \texttt{flexdashboard} ou \texttt{shiny}.
  \item Preparar uma apresentação executiva para um público não técnico.
  \item Documentar todo o processo de análise, incluindo código, decisões e fontes de dados.
\end{itemize}

\subsection{\textcolor{subsectionblue}{Recursos e Ferramentas}}
\begin{itemize}
  \item \textbf{Software:} R e RStudio.
  \item \textbf{Pacotes:} \texttt{rmarkdown}, \texttt{flexdashboard}, \texttt{shiny}, \texttt{plotly}, \texttt{leaflet}.
  \item \textbf{Ferramentas de versionamento:} Git e GitHub.
  \item \textbf{Datasets:} conjuntos de dados utilizados nos módulos anteriores.
\end{itemize}

\subsection{\textcolor{subsectionblue}{Estudo de Caso – Relatório e Dashboard em R}}
\textbf{Objetivo:} Demonstrar a criação de um relatório técnico e de um dashboard interativo.

\begin{lstlisting}[language=R]
# Relatório RMarkdown (chunk de código)
library(tidyverse)
dados <- read_csv("vendas.csv")

ggplot(dados, aes(x = mes, y = vendas, fill = regiao)) +
  geom_col(position = "dodge") +
  labs(title = "Vendas Mensais por Região",
       x = "Mês", y = "Total de Vendas")
\end{lstlisting}

Exemplo de cabeçalho e código para dashboard com flexdashboard:

\begin{lstlisting}
---
title: "Dashboard de Vendas"
output: flexdashboard::flex_dashboard
---

## Vendas por Região
```{r}
library(flexdashboard)
library(plotly)

grafico <- ggplot(dados, aes(x = regiao, y = vendas, fill = regiao)) +
  geom_bar(stat = "identity")
ggplotly(grafico)
\end{lstlisting}

\textbf{Notas para o Formador:} \begin{itemize} \item O código do relatório e do dashboard deve ser colocado em ficheiros \texttt{.Rmd} separados e executado no RStudio. \item No manual, o código é apenas exibido como exemplo. \item Incentivar a personalização do layout, cores e tipos de visualização. \item Demonstrar como publicar o dashboard no \texttt{shinyapps.io} ou partilhar o HTML gerado. \end{itemize}

% =========================
% Módulo 5
% =========================
\section{\textcolor{sectionred}{Módulo 5 – Research and Learning (6h)}}

\subsection{\textcolor{subsectionblue}{Objetivos}}
Ao final deste módulo, os formandos deverão ser capazes de:
\begin{itemize}
  \item Desenvolver competências de pesquisa contínua para se manterem atualizados nas áreas de Ciência de Dados, Machine Learning e Inteligência Artificial.  
  \item Avaliar criticamente novas bibliotecas, técnicas e metodologias antes de as adotar.  
  \item Integrar novas ferramentas e abordagens nos fluxos de trabalho existentes.  
  \item Partilhar conhecimento com a comunidade e com a equipa de trabalho.
\end{itemize}

\subsection{\textcolor{subsectionblue}{Conteúdos Desenvolvidos}}
\begin{itemize}
  \item \textbf{Fontes de informação e atualização:} repositórios de pacotes (CRAN, Bioconductor), blogs e comunidades (\texttt{R-bloggers}, Stack Overflow, Posit Community), publicações científicas (arXiv, IEEE, Nature Machine Intelligence).
  \item \textbf{Avaliação de novas ferramentas:} critérios de avaliação (desempenho, documentação, comunidade, manutenção), testes de integração em projetos piloto.
  \item \textbf{Integração no fluxo de trabalho:} adaptação de código e pipelines, automação de tarefas repetitivas.
  \item \textbf{Partilha de conhecimento:} documentação interna e externa, contribuição para projetos open-source.
\end{itemize}

\subsection{\textcolor{subsectionblue}{Atividades Práticas Detalhadas}}
\begin{itemize}
  \item Pesquisar e apresentar um pacote R recente, explicando as suas funcionalidades e casos de uso.
  \item Implementar um exemplo prático com uma técnica ou pacote recém-descoberto.
  \item Criar um documento comparativo entre duas abordagens para o mesmo problema.
  \item Publicar um pequeno tutorial ou exemplo de código num repositório GitHub.
\end{itemize}

\subsection{\textcolor{subsectionblue}{Recursos e Ferramentas}}
\begin{itemize}
  \item \textbf{Software:} R e RStudio.
  \item \textbf{Pacotes:} variáveis conforme a pesquisa (ex.: \texttt{tidyverse}, \texttt{data.table}, \texttt{lightgbm}).
  \item \textbf{Fontes de dados:} CRAN, GitHub, arXiv, APIs públicas.
\end{itemize}

\subsection{\textcolor{subsectionblue}{Estudo de Caso – Pesquisa e Integração de um Novo Pacote}}
\textbf{Objetivo:} Demonstrar como identificar, testar e integrar um novo pacote R num fluxo de trabalho.

\begin{lstlisting}[language=R]
# 1. Pesquisa de pacote no CRAN
# Exemplo: pacote 'data.table' para manipulação eficiente de dados

# 2. Instalação e carregamento
install.packages("data.table")
library(data.table)

# 3. Comparação de desempenho com dplyr
library(tidyverse)
dados <- as.data.frame(matrix(runif(1e6), ncol = 10))

# Usando dplyr
system.time({
  df_dplyr <- as_tibble(dados) %>%
    summarise(across(everything(), mean))
})

# Usando data.table
system.time({
  dt <- as.data.table(dados)
  dt_means <- dt[, lapply(.SD, mean)]
})

# 4. Integração no fluxo de trabalho
# Substituir operações lentas por equivalentes em data.table
# Documentar alterações e ganhos de desempenho

# 5. Partilha de resultados
# Criar um README com instruções e benchmarks
\end{lstlisting}

\textbf{Notas para o Formador:}
\begin{itemize}
  \item Incentivar os alunos a escolher pacotes relevantes para os seus interesses ou área de trabalho.
  \item Discutir critérios objetivos para adoção de novas ferramentas.
  \item Mostrar como documentar e comunicar resultados de testes à equipa.
\end{itemize}


% ================
=========
% Módulo 6
% =========================
\section{\textcolor{sectionred}{Módulo 6 – Ad-hoc Analysis (8h)}}

\subsection{\textcolor{subsectionblue}{Objetivos}}
Ao final deste módulo, os formandos deverão ser capazes de:
\begin{itemize}
  \item Compreender o papel das análises \textit{ad-hoc} na resposta rápida a questões específicas de negócio ou investigação.  
  \item Formular hipóteses e estruturar análises direcionadas para responder a perguntas concretas.  
  \item Selecionar e aplicar métodos estatísticos e de visualização adequados ao problema.  
  \item Comunicar resultados de forma clara e objetiva, focando nas conclusões mais relevantes.
\end{itemize}

\subsection{\textcolor{subsectionblue}{Conteúdos Desenvolvidos}}
\begin{itemize}
  \item \textbf{Definição e contexto de análises \textit{ad-hoc}} — diferença entre análises planeadas e \textit{ad-hoc}; quando e porquê utilizá-las.
  \item \textbf{Formulação de hipóteses e perguntas de negócio} — transformar questões vagas em hipóteses testáveis.
  \item \textbf{Seleção de métodos e ferramentas} — escolha de técnicas estatísticas, filtros e visualizações adequadas.
  \item \textbf{Execução rápida e validação de resultados} — garantir que a rapidez não compromete a qualidade.
  \item \textbf{Comunicação e entrega} — formatos de entrega: relatório breve, e-mail, apresentação curta.
\end{itemize}

\subsection{\textcolor{subsectionblue}{Atividades Práticas Detalhadas}}
\begin{itemize}
  \item Receber uma questão de negócio simulada e desenvolver uma análise \textit{ad-hoc} para respondê-la.
  \item Criar visualizações simples e eficazes para comunicar resultados.
  \item Aplicar um teste estatístico adequado ao problema apresentado.
  \item Documentar a análise num formato conciso e claro.
\end{itemize}

\subsection{\textcolor{subsectionblue}{Recursos e Ferramentas}}
\begin{itemize}
  \item \textbf{Software:} R e RStudio.
  \item \textbf{Pacotes:} \texttt{tidyverse}, \texttt{lubridate}, \texttt{ggplot2}, \texttt{broom}.
  \item \textbf{Fontes de dados:} datasets internos ou públicos simulando cenários reais.
\end{itemize}

\subsection{\textcolor{subsectionblue}{Estudo de Caso – Análise \textit{Ad-hoc} em R}}
\textbf{Objetivo:} Demonstrar como responder rapidamente a uma questão específica com dados disponíveis.

\begin{lstlisting}[language=R]
library(tidyverse)
library(lubridate)
library(broom)

# Cenário: O gestor quer saber se a média de vendas do último mês
# foi significativamente diferente do mês anterior.

# 1. Importar dados
dados <- read_csv("vendas.csv") %>%
  mutate(data = dmy(data))

# 2. Filtrar últimos dois meses
ultimo_mes <- max(month(dados$data))
dados_filtrados <- dados %>%
  filter(month(data) %in% c(ultimo_mes, ultimo_mes - 1))

# 3. Estatísticas descritivas
dados_filtrados %>%
  group_by(mes = month(data)) %>%
  summarise(media_vendas = mean(vendas, na.rm = TRUE),
            sd_vendas = sd(vendas, na.rm = TRUE))

# 4. Teste estatístico (t-test)
resultado_t <- t.test(vendas ~ month(data), data = dados_filtrados)
tidy(resultado_t)

# 5. Visualização rápida
ggplot(dados_filtrados, aes(x = factor(month(data)), y = vendas)) +
  geom_boxplot(fill = "skyblue") +
  labs(x = "Mês", y = "Vendas", title = "Comparação de Vendas entre Meses")

# 6. Conclusão
# Se p-valor < 0.05, há diferença estatisticamente significativa.
\end{lstlisting}

\textbf{Notas para o Formador:}
\begin{itemize}
  \item Este exemplo mostra como estruturar rapidamente uma análise desde a importação de dados até à conclusão.
  \item Incentivar os alunos a adaptar o código a outros tipos de questões \textit{ad-hoc}.
  \item Discutir a importância de comunicar apenas o essencial em análises urgentes.
\end{itemize}


% =========================
% Módulo 7
% =========================
\section{\textcolor{sectionred}{Módulo 7 – Tooling and Infrastructure (6h)}}

\subsection{\textcolor{subsectionblue}{Objetivos}}
Ao final deste módulo, os formandos deverão ser capazes de:
\begin{itemize}
  \item Configurar e manter ambientes de desenvolvimento eficientes para projetos de Ciência de Dados.  
  \item Automatizar tarefas e fluxos de trabalho para aumentar a produtividade.  
  \item Utilizar ferramentas de controlo de versões para gerir código e colaborar em equipa.  
  \item Integrar pipelines de dados e modelos em ambientes de produção.
\end{itemize}

\subsection{\textcolor{subsectionblue}{Conteúdos Desenvolvidos}}
\begin{itemize}
  \item \textbf{Organização de projetos:} estrutura de pastas e ficheiros; nomeação consistente de ficheiros e scripts.
  \item \textbf{Gestão de dependências:} uso do \texttt{renv} para isolar ambientes; ficheiros \texttt{DESCRIPTION} e \texttt{requirements.txt}.
  \item \textbf{Controlo de versões:} Git básico (commits, branches, merges); plataformas de colaboração (GitHub, GitLab).
  \item \textbf{Automação e pipelines:} scripts agendados (cron jobs, tasks); integração contínua (CI) e entrega contínua (CD).
\end{itemize}

\subsection{\textcolor{subsectionblue}{Atividades Práticas Detalhadas}}
\begin{itemize}
  \item Criar a estrutura de um projeto de Ciência de Dados com pastas e scripts organizados.
  \item Configurar um ambiente isolado com \texttt{renv} e instalar pacotes necessários.
  \item Criar um repositório Git e praticar operações básicas (commit, branch, merge).
  \item Configurar um script para ser executado automaticamente (agendamento).
\end{itemize}

\subsection{\textcolor{subsectionblue}{Recursos e Ferramentas}}
\begin{itemize}
  \item \textbf{Software:} R, RStudio, Git.
  \item \textbf{Pacotes:} \texttt{renv}, \texttt{targets}, \texttt{usethis}.
  \item \textbf{Plataformas:} GitHub, GitLab.
\end{itemize}

\subsection{\textcolor{subsectionblue}{Estudo de Caso – Estruturação e Automação de um Projeto}}
\textbf{Objetivo:} Demonstrar como criar um projeto organizado, gerir dependências e automatizar tarefas.

\begin{lstlisting}[language=R]
# 1. Criar estrutura de projeto
usethis::create_project("projeto_vendas")
dir.create("data_raw")
dir.create("data_processed")
dir.create("scripts")
dir.create("reports")

# 2. Iniciar controlo de versões
usethis::use_git()

# 3. Configurar ambiente isolado
install.packages("renv")
renv::init()

# 4. Script de importação e limpeza (scripts/01_importacao.R)
library(tidyverse)
dados <- read_csv("data_raw/vendas.csv") %>%
  janitor::clean_names() %>%
  filter(!is.na(valor_venda))
write_csv(dados, "data_processed/vendas_limpo.csv")

# 5. Automatizar execução diária (exemplo em sistema Unix com cron)
# Abrir crontab: crontab -e
# Adicionar linha para executar script às 2h da manhã:
# 0 2 * * * Rscript /caminho/projeto_vendas/scripts/01_importacao.R

# 6. Versionar alterações e sincronizar com GitHub
# git add .
# git commit -m "Estrutura inicial e script de importação"
# git push origin main
\end{lstlisting}

\textbf{Notas para o Formador:}
\begin{itemize}
  \item Este estudo de caso mostra como combinar organização, gestão de dependências, controlo de versões e automação.
  \item Incentivar os alunos a adaptar a estrutura e scripts às necessidades dos seus próprios projetos.
  \item Discutir a importância de manter um README atualizado com instruções de execução.
\end{itemize}


% =========================
% Módulo 8
% =========================
\section{\textcolor{sectionred}{Módulo 8 – Deployment e Manutenção de Modelos (6h)}}

\subsection{\textcolor{subsectionblue}{Objetivos}}
Ao final deste módulo, os formandos deverão ser capazes de:
\begin{itemize}
  \item Compreender o processo de disponibilização de modelos de Machine Learning em ambientes de produção.  
  \item Implementar APIs para servir previsões de modelos em tempo real ou em lote.  
  \item Monitorizar o desempenho de modelos em produção e detetar \textit{data drift} ou degradação de performance.  
  \item Realizar manutenção e re-treino periódico de modelos para garantir relevância e precisão.  
  \item Documentar e versionar modelos para rastreabilidade e auditoria.
\end{itemize}

\subsection{\textcolor{subsectionblue}{Conteúdos Desenvolvidos}}
\begin{itemize}
  \item \textbf{Conceitos de deployment:} diferença entre ambientes de desenvolvimento, teste e produção; modos de disponibilização (batch, tempo real, \textit{edge deployment}).
  \item \textbf{Servir modelos via API:} uso do pacote \texttt{plumber} para criar APIs REST em R; segurança e autenticação de APIs.
  \item \textbf{Monitorização de modelos:} métricas de desempenho em produção; deteção de \textit{data drift} e \textit{concept drift}.
  \item \textbf{Manutenção e re-treino:} estratégias de re-treino (agendado, baseado em eventos); automação de pipelines de re-treino.
  \item \textbf{Versionamento e documentação de modelos:} armazenamento de modelos com \texttt{pins} ou \texttt{modeltime}; registo de alterações e auditoria.
\end{itemize}

\subsection{\textcolor{subsectionblue}{Atividades Práticas Detalhadas}}
\begin{itemize}
  \item Criar uma API com \texttt{plumber} para servir previsões de um modelo treinado.
  \item Implementar um script de monitorização que regista métricas de desempenho diariamente.
  \item Simular \textit{data drift} e executar re-treino automático.
  \item Documentar o processo de deployment e manutenção num relatório técnico.
\end{itemize}

\subsection{\textcolor{subsectionblue}{Recursos e Ferramentas}}
\begin{itemize}
  \item \textbf{Software:} R e RStudio.
  \item \textbf{Pacotes:} \texttt{plumber}, \texttt{pins}, \texttt{modeltime}, \texttt{tidyverse}.
  \item \textbf{Plataformas:} Servidores locais, \texttt{shinyapps.io}, Docker.
\end{itemize}

\subsection{\textcolor{subsectionblue}{Estudo de Caso – API de Previsões com \texttt{plumber}}}
\textbf{Objetivo:} Demonstrar como disponibilizar um modelo treinado via API REST.

\begin{lstlisting}[language=R]
# 1. Treinar e guardar modelo
library(tidyverse)
library(caret)
dados <- read_csv("dados_clientes.csv")

modelo <- train(churn ~ ., data = dados, method = "glm", family = "binomial")
saveRDS(modelo, "modelo_churn.rds")

# 2. Criar API com plumber (ficheiro: api.R)
library(plumber)

#* @apiTitle API de Previsão de Churn
#* @param idade:int Idade do cliente
#* @param rendimento:double Rendimento anual
#* @post /prever
function(idade, rendimento){
  modelo <- readRDS("modelo_churn.rds")
  novo_dado <- data.frame(idade = as.integer(idade),
                          rendimento = as.numeric(rendimento))
  prob <- predict(modelo, novo_dado, type = "prob")[,2]
  list(probabilidade_churn = prob)
}

# 3. Executar API
# plumber::pr("api.R") %>% pr_run(port = 8000)

# 4. Testar API
# POST para http://localhost:8000/prever com JSON:
# {"idade": 35, "rendimento": 45000}
\end{lstlisting}

\textbf{Notas para o Formador:}
\begin{itemize}
  \item Este exemplo mostra um fluxo simples de deployment com \texttt{plumber}.
  \item Incentivar os alunos a adicionar autenticação e logging à API.
  \item Discutir como integrar esta API num sistema maior (ex.: dashboard, aplicação web).
\end{itemize}


% =========================
% Módulo 9
% =========================
\section{\textcolor{sectionred}{Módulo 9 – Avaliação (4h)}}

\subsection{\textcolor{subsectionblue}{Objetivos}}
Ao final deste módulo, os formandos deverão ser capazes de:
\begin{itemize}
  \item Definir critérios e instrumentos de avaliação adequados aos objetivos da formação.  
  \item Aplicar diferentes tipos de avaliação (diagnóstica, formativa e sumativa) ao longo do curso.  
  \item Fornecer feedback construtivo e orientado à melhoria contínua.  
  \item Utilizar ferramentas digitais para recolher, registar e analisar resultados de avaliação.  
  \item Documentar o processo de avaliação para garantir transparência e rastreabilidade.
\end{itemize}

\subsection{\textcolor{subsectionblue}{Conteúdos Desenvolvidos}}
\begin{itemize}
  \item \textbf{Tipos de avaliação:} diagnóstica, formativa e sumativa.
  \item \textbf{Critérios e indicadores de desempenho:} clareza, objetividade e alinhamento com os objetivos do curso.
  \item \textbf{Instrumentos de avaliação:} grelhas de avaliação, rubricas, testes, apresentações; ferramentas digitais (Google Forms, Microsoft Forms, Moodle).
  \item \textbf{Feedback construtivo:} estrutura “pontos fortes – pontos a melhorar – sugestões”; técnicas de comunicação assertiva.
  \item \textbf{Registo e documentação:} armazenamento seguro de resultados; relatórios de avaliação e atas de reuniões de feedback.
\end{itemize}

\subsection{\textcolor{subsectionblue}{Atividades Práticas Detalhadas}}
\begin{itemize}
  \item Criar uma grelha de avaliação para um projeto final, com critérios e pesos definidos.
  \item Aplicar a grelha a um trabalho exemplo e calcular a nota final.
  \item Elaborar um relatório de feedback com base nos resultados.
  \item Simular a utilização de uma ferramenta digital para recolher e analisar resultados.
\end{itemize}

\subsection{\textcolor{subsectionblue}{Recursos e Ferramentas}}
\begin{itemize}
  \item \textbf{Software:} R e RStudio, Google Forms, Microsoft Forms, Moodle.
  \item \textbf{Pacotes R:} \texttt{dplyr}, \texttt{readr}, \texttt{knitr}.
  \item \textbf{Documentos:} modelos de grelhas e rubricas.
\end{itemize}

\subsection{\textcolor{subsectionblue}{Estudo de Caso – Grelha de Avaliação em R}}
\textbf{Objetivo:} Demonstrar como criar e aplicar uma grelha de avaliação com cálculo automático da nota final.

\begin{lstlisting}[language=R]
library(dplyr)

# 1. Definir critérios e pesos
criterios <- data.frame(
  criterio = c("Qualidade Técnica", "Criatividade", "Comunicação", "Documentação"),
  peso = c(0.4, 0.2, 0.2, 0.2),
  pontuacao = c(4, 5, 4, 3) # exemplo de avaliação (1 a 5)
)

# 2. Calcular nota final (escala 0-20)
criterios <- criterios %>%
  mutate(resultado = peso * pontuacao)

nota_final <- sum(criterios$resultado) / sum(criterios$peso) * 4 # 5 pontos = 20 valores

# 3. Mostrar resultados
criterios
nota_final

# 4. Exportar relatório simples
library(knitr)
kable(criterios, col.names = c("Critério", "Peso", "Pontuação", "Resultado"))
cat("Nota Final:", round(nota_final, 2), "/ 20")
\end{lstlisting}

\textbf{Notas para o Formador:}
\begin{itemize}
  \item Este exemplo mostra como usar R para automatizar o cálculo de notas com base numa grelha.
  \item Incentivar os alunos a adaptar critérios e pesos ao tipo de projeto.
  \item Discutir a importância de manter registos claros e acessíveis para auditoria.
\end{itemize}


\end{document}
